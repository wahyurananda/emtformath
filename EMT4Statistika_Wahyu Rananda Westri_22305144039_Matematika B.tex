\documentclass[a4paper,10pt]{article}
\usepackage{eumat}

\begin{document}
\begin{eulernotebook}
\eulerheading{EMT untuk Statistika}
\begin{eulercomment}
Dalam buku catatan ini, kami menunjukkan plot statistik utama, uji,
dan distribusi dalam Euler.

Mari kita mulai dengan beberapa statistik deskriptif. Ini bukan
pengantar statistika. Jadi, Anda mungkin memerlukan beberapa
pengetahuan dasar untuk memahami rincian tersebut.

Anggaplah pengukuran berikut. Kami ingin menghitung nilai rata-rata
dan deviasi standar yang diukur.
\end{eulercomment}
\begin{eulerprompt}
>M=[1000,1004,998,997,1002,1001,998,1004,998,997]; ...
>median(M), mean(M), dev(M),
\end{eulerprompt}
\begin{euleroutput}
  999
  999.9
  2.72641400622
\end{euleroutput}
\begin{eulercomment}
Kita dapat membuat plot diagram kotak dan garis (box-and-whiskers)
untuk data ini. Dalam kasus kita, tidak ada data yang berada di luar
jangkauan (outliers).
\end{eulercomment}
\begin{eulerprompt}
>aspect(1.75); boxplot(M):
\end{eulerprompt}
\eulerimg{15}{images/EMT4Statistika_Wahyu Rananda Westri_22305144039_Matematika B-001.png}
\begin{eulercomment}
Kami menghitung probabilitas bahwa suatu nilai lebih besar dari 1005,
dengan asumsi nilai yang diukur berasal dari distribusi normal.

Semua fungsi distribusi dalam Euler diakhiri dengan ...dis dan
menghitung distribusi probabilitas kumulatif (CPD).

\end{eulercomment}
\begin{eulerformula}
\[
\text{normaldis(x,m,d)}=\int_{-\infty}^x \frac{1}{d\sqrt{2\pi}}e^{-\frac{1}{2}(\frac{t-m}{d})^2}\ dt.
\]
\end{eulerformula}
\begin{eulercomment}
Kami mencetak hasilnya dalam bentuk persen dengan akurasi dua digit
desimal menggunakan fungsi print.
\end{eulercomment}
\begin{eulerprompt}
>print((1-normaldis(1005,mean(M),dev(M)))*100,2,unit=" %")
\end{eulerprompt}
\begin{euleroutput}
        3.07 %
\end{euleroutput}
\begin{eulercomment}
Untuk contoh berikutnya, kami mengasumsikan jumlah pria dalam rentang
ukuran yang diberikan sebagai berikut.
\end{eulercomment}
\begin{eulerprompt}
>r=155.5:4:187.5; v=[22,71,136,169,139,71,32,8];
\end{eulerprompt}
\begin{eulercomment}
Berikut adalah plot dari distribusinya.
\end{eulercomment}
\begin{eulerprompt}
>plot2d(r,v,a=150,b=200,c=0,d=190,bar=1,style="\(\backslash\)/"):
\end{eulerprompt}
\eulerimg{15}{images/EMT4Statistika_Wahyu Rananda Westri_22305144039_Matematika B-003.png}
\begin{eulercomment}
Kita dapat memasukkan data mentah seperti ini ke dalam tabel.

Tabel adalah metode untuk menyimpan data statistik. Tabel kita
seharusnya memiliki tiga kolom: Awal rentang, akhir rentang, jumlah
pria dalam rentang tersebut.

Tabel dapat dicetak dengan judul. Kami menggunakan vektor string untuk
menetapkan judul kolom.
\end{eulercomment}
\begin{eulerprompt}
>T:=r[1:8]' | r[2:9]' | v'; writetable(T,labc=["BB","BA","Frek"])
\end{eulerprompt}
\begin{euleroutput}
          BB        BA      Frek
       155.5     159.5        22
       159.5     163.5        71
       163.5     167.5       136
       167.5     171.5       169
       171.5     175.5       139
       175.5     179.5        71
       179.5     183.5        32
       183.5     187.5         8
\end{euleroutput}
\begin{eulercomment}
Jika kita membutuhkan nilai rata-rata dan statistik lainnya dari
ukuran tersebut, kita perlu menghitung titik tengah dari rentang
tersebut. Kita dapat menggunakan dua kolom pertama dari tabel kita
untuk ini.

Simbol "\textbar{}" digunakan untuk memisahkan kolom, fungsi "writetable"
digunakan untuk menulis tabel, dengan opsi "labc" digunakan untuk
menentukan judul kolom.
\end{eulercomment}
\begin{eulerprompt}
>(T[,1]+T[,2])/2 // nilai tengah dari tiap interval
\end{eulerprompt}
\begin{euleroutput}
          157.5 
          161.5 
          165.5 
          169.5 
          173.5 
          177.5 
          181.5 
          185.5 
\end{euleroutput}
\begin{eulercomment}
Namun, lebih mudah untuk menjumlahkan rentang tersebut dengan vektor
[1/2, 1/2].
\end{eulercomment}
\begin{eulerprompt}
>M=fold(r,[0.5,0.5])
\end{eulerprompt}
\begin{euleroutput}
  [157.5,  161.5,  165.5,  169.5,  173.5,  177.5,  181.5,  185.5]
\end{euleroutput}
\begin{eulercomment}
Sekarang kita dapat menghitung nilai rata-rata dan deviasi dari sampel
dengan frekuensi yang diberikan.
\end{eulercomment}
\begin{eulerprompt}
>\{m,d\}=meandev(M,v); m, d,
\end{eulerprompt}
\begin{euleroutput}
  169.901234568
  5.98912964449
\end{euleroutput}
\begin{eulercomment}
Mari tambahkan distribusi normal dari nilai-nilai tersebut ke plot
batang di atas. Rumus untuk distribusi normal dengan rata-rata m dan
deviasi standar d adalah:

\end{eulercomment}
\begin{eulerformula}
\[
y=\frac{1}{d\sqrt{2\pi}}e^{\frac{-(x-m)^2}{2d^2}}.
\]
\end{eulerformula}
\begin{eulercomment}
Karena nilai-nilainya berada di antara 0 dan 1, untuk memplotnya pada
diagram batang, nilai tersebut harus dikalikan dengan 4 kali jumlah
total data.
\end{eulercomment}
\begin{eulerprompt}
>plot2d("qnormal(x,m,d)*sum(v)*4", ...
>  xmin=min(r),xmax=max(r),thickness=3,add=1):
\end{eulerprompt}
\eulerimg{15}{images/EMT4Statistika_Wahyu Rananda Westri_22305144039_Matematika B-005.png}
\eulersubheading{Contoh Soal}
\begin{eulercomment}
1) Tentukan median, mean, deviasi, dan distribusi kumulatif normal
(CDF) dari data tersebut.

\end{eulercomment}
\begin{eulerformula}
\[
N=[199,200,201,202,203,204,205,205,206,207,208,208,209]
\]
\end{eulerformula}
\begin{eulercomment}
\end{eulercomment}
\begin{eulerprompt}
>N=[199, 200, 201,202,203,204,205,205,206,207,208,208,209]
\end{eulerprompt}
\begin{euleroutput}
  [199,  200,  201,  202,  203,  204,  205,  205,  206,  207,  208,  208,
  209]
\end{euleroutput}
\begin{eulerprompt}
>median(N), mean(N), dev(N),
\end{eulerprompt}
\begin{euleroutput}
  205
  204.384615385
  3.22847904176
\end{euleroutput}
\begin{eulerprompt}
>aspect(1.5); boxplot(N):
\end{eulerprompt}
\eulerimg{17}{images/EMT4Statistika_Wahyu Rananda Westri_22305144039_Matematika B-007.png}
\begin{eulerprompt}
>print((1-normaldis(199,mean(N),dev(N)))*100,2,unit=" %")
\end{eulerprompt}
\begin{euleroutput}
       95.23 %
\end{euleroutput}
\begin{eulercomment}
2) Buatlah plot dari jumlah siswa dalam rentang yang diberikan sebagai
berikut.
\end{eulercomment}
\begin{eulerprompt}
>r=50:5:100; v=[8,1,3,2,5,6,7,2,3,5];
>plot2d(r,v,a=40,b=100,c=0,d=10,bar=1,style="\(\backslash\)/"):
\end{eulerprompt}
\eulerimg{17}{images/EMT4Statistika_Wahyu Rananda Westri_22305144039_Matematika B-008.png}
\eulerheading{Tabel}
\begin{eulercomment}
Di direktori buku catatan ini, Anda akan menemukan sebuah file berisi
tabel. Data tersebut mewakili hasil dari sebuah survei. Berikut adalah
empat baris pertama dari file tersebut. Data ini berasal dari sebuah
buku online Jerman "Einführung in die Statistik mit R" oleh A. Handl.
\end{eulercomment}
\begin{eulerprompt}
>printfile("table.dat",4); //function printfile (filename, lines)
\end{eulerprompt}
\begin{euleroutput}
  Could not open the file
  table.dat
  for reading!
  Try "trace errors" to inspect local variables after errors.
  printfile:
      open(filename,"r");
\end{euleroutput}
\begin{eulercomment}
Tabel tersebut berisi 7 kolom angka atau token (string). Kami ingin
membaca tabel dari file tersebut. Pertama, kami akan menggunakan
terjemahan kami sendiri untuk token-token tersebut.

Untuk melakukan ini, kami mendefinisikan set token-token tersebut.
Fungsi strtokens() mendapatkan vektor string token dari sebuah string
yang diberikan.
\end{eulercomment}
\begin{eulerprompt}
>mf:=["m","f"]; yn:=["y","n"]; ev:=strtokens("g vg m b vb");
\end{eulerprompt}
\begin{eulercomment}
Sekarang kita akan membaca tabel dengan terjemahan ini.

Argumen tok2, tok4, dan sebagainya adalah terjemahan dari kolom-kolom
tabel. Argumen-argumen ini tidak ada dalam daftar parameter
readtable(), sehingga Anda perlu menyediakannya dengan ":=".
\end{eulercomment}
\begin{eulerprompt}
>\{MT,hd\}=readtable("table.dat",tok2:=mf,tok4:=yn,tok5:=ev,tok7:=yn);
\end{eulerprompt}
\begin{euleroutput}
  Could not open the file
  table.dat
  for reading!
  Try "trace errors" to inspect local variables after errors.
  readtable:
      if filename!=none then open(filename,"r"); endif;
\end{euleroutput}
\begin{eulerprompt}
>load over statistics;
\end{eulerprompt}
\begin{eulercomment}
Untuk mencetak, kita perlu menentukan set token yang sama. Kita hanya
akan mencetak empat baris pertama.
\end{eulercomment}
\begin{eulerprompt}
>writetable(MT[1:10],labc=hd,wc=5,tok2:=mf,tok4:=yn,tok5:=ev,tok7:=yn);
\end{eulerprompt}
\begin{euleroutput}
  MT is not a variable!
  Error in:
  writetable(MT[1:10],labc=hd,wc=5,tok2:=mf,tok4:=yn,tok5:=ev,to ...
                     ^
\end{euleroutput}
\begin{eulercomment}
Titik "." mewakili nilai yang tidak tersedia.

Jika kita tidak ingin menentukan token untuk terjemahan sebelumnya,
kita hanya perlu menentukan kolom-kolom yang berisi token, bukan
angka..
\end{eulercomment}
\begin{eulerprompt}
>ctok=[2,4,5,7]; \{MT,hd,tok\}=readtable("table.dat",ctok=ctok);
\end{eulerprompt}
\begin{euleroutput}
  Could not open the file
  table.dat
  for reading!
  Try "trace errors" to inspect local variables after errors.
  readtable:
      if filename!=none then open(filename,"r"); endif;
\end{euleroutput}
\begin{eulercomment}
Fungsi readtable() sekarang mengembalikan sebuah set token.
\end{eulercomment}
\begin{eulerprompt}
>tok
\end{eulerprompt}
\begin{euleroutput}
  Variable tok not found!
  Error in:
  tok ...
     ^
\end{euleroutput}
\begin{eulercomment}
Tabel tersebut berisi entri dari file dengan token diterjemahkan
menjadi angka.

String khusus NA="." diartikan sebagai "Not Available" (Tidak
Tersedia), dan diubah menjadi NAN (bukan angka) dalam tabel.
Terjemahan ini dapat diubah dengan parameter NA dan NAval.
\end{eulercomment}
\begin{eulerprompt}
>MT[1]
\end{eulerprompt}
\begin{euleroutput}
  MT is not a variable!
  Error in:
  MT[1] ...
       ^
\end{euleroutput}
\begin{eulercomment}
Berikut adalah konten tabel dengan angka yang tidak diterjemahkan.
\end{eulercomment}
\begin{eulerprompt}
>writetable(MT,wc=5)
\end{eulerprompt}
\begin{euleroutput}
  Variable or function MT not found.
  Error in:
  writetable(MT,wc=5) ...
               ^
\end{euleroutput}
\begin{eulercomment}
Untuk kenyamanan, Anda dapat menyimpan output dari readtable() ke
dalam sebuah daftar (list).
\end{eulercomment}
\begin{eulerprompt}
>Table=\{\{readtable("table.dat",ctok=ctok)\}\};
\end{eulerprompt}
\begin{euleroutput}
  Could not open the file
  table.dat
  for reading!
  Try "trace errors" to inspect local variables after errors.
  readtable:
      if filename!=none then open(filename,"r"); endif;
\end{euleroutput}
\begin{eulercomment}
Dengan menggunakan kolom-kolom token yang sama dan token yang dibaca
dari file, kita dapat mencetak tabel tersebut. Kita dapat menentukan
ctok, tok, dll., atau menggunakan daftar (list) Table.
\end{eulercomment}
\begin{eulerprompt}
>writetable(Table,ctok=ctok,wc=5);
\end{eulerprompt}
\begin{euleroutput}
  Variable or function Table not found.
  Error in:
  writetable(Table,ctok=ctok,wc=5); ...
                  ^
\end{euleroutput}
\begin{eulercomment}
Fungsi tablecol() mengembalikan nilai-nilai dari kolom-kolom tabel,
melewati baris-baris dengan nilai NAN ("." dalam file), serta
indeks-indeks kolom yang berisi nilai-nilai tersebut.
\end{eulercomment}
\begin{eulerprompt}
>\{c,i\}=tablecol(MT,[5,6]);
\end{eulerprompt}
\begin{euleroutput}
  Variable or function MT not found.
  Error in:
  \{c,i\}=tablecol(MT,[5,6]); ...
                   ^
\end{euleroutput}
\begin{eulercomment}
Kita dapat menggunakan ini untuk mengekstrak kolom-kolom dari tabel
untuk membuat tabel baru.
\end{eulercomment}
\begin{eulerprompt}
>j=[1,5,6]; writetable(MT[i,j],labc=hd[j],ctok=[2],tok=tok)
\end{eulerprompt}
\begin{euleroutput}
  Variable or function i not found.
  Error in:
  j=[1,5,6]; writetable(MT[i,j],labc=hd[j],ctok=[2],tok=tok) ...
                            ^
\end{euleroutput}
\begin{eulercomment}
Tentu saja, dalam hal ini, kita perlu mengekstrak tabel itu sendiri
dari daftar (list) Table.
\end{eulercomment}
\begin{eulerprompt}
>MT=Table[1];
\end{eulerprompt}
\begin{euleroutput}
  Table is not a variable!
  Error in:
  MT=Table[1]; ...
             ^
\end{euleroutput}
\begin{eulercomment}
Tentu saja, kita juga dapat menggunakannya untuk menentukan nilai
rata-rata dari suatu kolom atau nilai statistik lainnya.
\end{eulercomment}
\begin{eulerprompt}
>mean(tablecol(MT,6))
\end{eulerprompt}
\begin{euleroutput}
  Variable or function MT not found.
  Error in:
  mean(tablecol(MT,6)) ...
                  ^
\end{euleroutput}
\begin{eulercomment}
Fungsi getstatistics() mengembalikan elemen-elemen dalam bentuk
vektor, beserta jumlah kemunculannya. Kita menggunakannya untuk nilai
"m" dan "f" dalam kolom kedua tabel kita.
\end{eulercomment}
\begin{eulerprompt}
>\{xu,count\}=getstatistics(tablecol(MT,2)); xu, count,
\end{eulerprompt}
\begin{euleroutput}
  Variable or function MT not found.
  Error in:
  \{xu,count\}=getstatistics(tablecol(MT,2)); xu, count, ...
                                      ^
\end{euleroutput}
\begin{eulercomment}
Kita dapat mencetak hasilnya dalam bentuk tabel baru.
\end{eulercomment}
\begin{eulerprompt}
>writetable(count',labr=tok[xu])
\end{eulerprompt}
\begin{euleroutput}
  Variable count not found!
  Error in:
  writetable(count',labr=tok[xu]) ...
                   ^
\end{euleroutput}
\begin{eulercomment}
Fungsi selecttable() mengembalikan tabel baru dengan nilai-nilai dalam
satu kolom yang dipilih dari vektor indeks. Pertama, kita mencari
indeks dua nilai dalam tabel token kita.

Catatan tambahan:\\
Fungsi indexof(v, x) berarti bahwa mencari x dalam vektor v.
\end{eulercomment}
\begin{eulerprompt}
>v:=indexof(tok,["g","vg"])
\end{eulerprompt}
\begin{euleroutput}
  Variable or function tok not found.
  Error in:
  v:=indexof(tok,["g","vg"]) ...
                ^
\end{euleroutput}
\begin{eulercomment}
Sekarang kita dapat memilih baris-baris tabel yang memiliki salah satu
dari nilai-nilai dalam vektor v di kolom kelima mereka.
\end{eulercomment}
\begin{eulerprompt}
>MT1:=MT[selectrows(MT,5,v)]; i:=sortedrows(MT1,5);
\end{eulerprompt}
\begin{euleroutput}
  Variable or function MT not found.
  Error in:
  MT1:=MT[selectrows(MT,5,v)]; i:=sortedrows(MT1,5); ...
                       ^
\end{euleroutput}
\begin{eulercomment}
Sekarang kita dapat mencetak tabel dengan nilai-nilai yang diekstrak
dan diurutkan dalam kolom kelima.
\end{eulercomment}
\begin{eulerprompt}
>writetable(MT1[i],labc=hd,ctok=ctok,tok=tok,wc=7);
\end{eulerprompt}
\begin{euleroutput}
  Variable or function i not found.
  Error in:
  writetable(MT1[i],labc=hd,ctok=ctok,tok=tok,wc=7); ...
                  ^
\end{euleroutput}
\begin{eulercomment}
Untuk statistik selanjutnya, kita ingin mengaitkan dua kolom dari
tabel. Jadi, kita akan mengekstrak kolom 2 dan 4, lalu mengurutkan
tabelnya.
\end{eulercomment}
\begin{eulerprompt}
>i=sortedrows(MT,[2,4]);  ...
>  writetable(tablecol(MT[i],[2,4])',ctok=[1,2],tok=tok)//Tabel diurutkan secara leksikografis.
\end{eulerprompt}
\begin{euleroutput}
  Variable or function MT not found.
  Error in:
  i=sortedrows(MT,[2,4]);    writetable(tablecol(MT[i],[2,4])',c ...
                 ^
\end{euleroutput}
\begin{eulercomment}
Dengan menggunakan fungsi getstatistics(), kita dapat menghubungkan
jumlah kemunculan dalam dua kolom tabel satu sama lain.
\end{eulercomment}
\begin{eulerprompt}
>MT24=tablecol(MT,[2,4]); ...
>\{xu1,xu2,count\}=getstatistics(MT24[1],MT24[2]); ...
>writetable(count,labr=tok[xu1],labc=tok[xu2])
\end{eulerprompt}
\begin{euleroutput}
  Variable or function MT not found.
  Error in:
  MT24=tablecol(MT,[2,4]); \{xu1,xu2,count\}=getstatistics(MT24[1] ...
                  ^
\end{euleroutput}
\begin{eulercomment}
Sebuah tabel dapat ditulis ke dalam sebuah file.
\end{eulercomment}
\begin{eulerprompt}
>filename="test.dat"; ...
>writetable(count,labr=tok[xu1],labc=tok[xu2],file=filename);
\end{eulerprompt}
\begin{euleroutput}
  Variable or function count not found.
  Error in:
  filename="test.dat"; writetable(count,labr=tok[xu1],labc=tok[x ...
                                       ^
\end{euleroutput}
\begin{eulercomment}
Kemudian kita dapat membaca tabel dari file tersebut.
\end{eulercomment}
\begin{eulerprompt}
>\{MT2,hd,tok2,hdr\}=readtable(filename,>clabs,>rlabs); ...
>writetable(MT2,labr=hdr,labc=hd)
\end{eulerprompt}
\begin{euleroutput}
  Could not open the file
  test.dat
  for reading!
  Try "trace errors" to inspect local variables after errors.
  readtable:
      if filename!=none then open(filename,"r"); endif;
\end{euleroutput}
\begin{eulercomment}
Dan menghapus file tersebut.
\end{eulercomment}
\begin{eulerprompt}
>fileremove(filename);
\end{eulerprompt}
\eulerheading{Distribusi}
\begin{eulercomment}
Dengan plot2d, ada metode yang sangat mudah untuk memplot distribusi
data eksperimental.
\end{eulercomment}
\begin{eulerprompt}
>p=normal(1,1000); //1000 sampel acak yang terdistribusi normal p
>plot2d(p,distribution=20,style="\(\backslash\)/"); // plot sampel acak p
>plot2d("qnormal(x,0,1)",add=1): // tambahkan plot distribusi normal standar
\end{eulerprompt}
\eulerimg{17}{images/EMT4Statistika_Wahyu Rananda Westri_22305144039_Matematika B-009.png}
\begin{eulercomment}
Harap perhatikan perbedaan antara diagram batang (sampel) dan kurva
normal (distribusi sebenarnya). Silakan masukkan kembali tiga perintah
tersebut untuk melihat hasil pengambilan sampel yang lain.
\end{eulercomment}
\begin{eulerprompt}
>p=normal(1,1000); 
>plot2d(p,distribution=20,style="\(\backslash\)/"); 
>plot2d("qnormal(x,0,1)",add=1): 
\end{eulerprompt}
\eulerimg{17}{images/EMT4Statistika_Wahyu Rananda Westri_22305144039_Matematika B-010.png}
\begin{eulercomment}
Berikut adalah perbandingan dari 10 simulasi 1000 nilai yang
terdistribusi secara normal menggunakan diagram kotak (box plot). Plot
ini menunjukkan median, kuartil 25\% dan 75\%, nilai minimal dan
maksimal, serta nilai-nilai yang berada di luar jangkauan (outliers).
\end{eulercomment}
\begin{eulerprompt}
>p=normal(10,1000); boxplot(p):
\end{eulerprompt}
\eulerimg{17}{images/EMT4Statistika_Wahyu Rananda Westri_22305144039_Matematika B-011.png}
\begin{eulercomment}
Untuk menghasilkan bilangan bulat acak, Euler memiliki fungsi
intrandom. Mari kita simulasi lemparan dadu dan plot distribusinya.

Kita akan menggunakan fungsi getmultiplicities(v, x) yang menghitung
seberapa sering elemen-elemen dari v muncul dalam x. Kemudian, kita
akan memplot hasilnya menggunakan fungsi columnsplot().
\end{eulercomment}
\begin{eulerprompt}
>k=intrandom(1,6000,6);  ...
>columnsplot(getmultiplicities(1:6,k));  ...
>ygrid(1000,color=red):
\end{eulerprompt}
\eulerimg{17}{images/EMT4Statistika_Wahyu Rananda Westri_22305144039_Matematika B-012.png}
\begin{eulercomment}
Catatan tambahan :\\
intrandom(n, m, k): Matriks dari variabel acak\\
getmultiplicities : Menghitung seberapa sering elemen-elemen dari x
muncul dalam y.

Meskipun intrandom(n, m, k) mengembalikan bilangan bulat yang
terdistribusi seragam dari 1 hingga k, kita juga dapat menggunakan
distribusi bilangan bulat lainnya dengan menggunakan randpint().

Pada contoh berikut, probabilitas untuk 1, 2, 3 adalah 0,4, 0,1, 0,5
berturut-turut.
\end{eulercomment}
\begin{eulerprompt}
>randpint(1,1000,[0.4,0.1,0.5]); getmultiplicities(1:3,%)
\end{eulerprompt}
\begin{euleroutput}
  [394,  107,  499]
\end{euleroutput}
\begin{eulercomment}
Euler dapat menghasilkan nilai acak dari berbagai distribusi. Silakan
lihat dokumentasi (reference) untuk informasi lebih lanjut.

Sebagai contoh, kita akan mencoba distribusi eksponensial. Sebuah
variabel acak kontinu X dikatakan memiliki distribusi eksponensial
jika PDF (Probability Density Function) nya diberikan oleh:\\
\end{eulercomment}
\begin{eulerformula}
\[
f_X(x)=\lambda e^{-\lambda x},\quad x>0,\quad \lambda>0,
\]
\end{eulerformula}
\begin{eulercomment}
dengan parameter\\
\end{eulercomment}
\begin{eulerformula}
\[
\lambda=\frac{1}{\mu},\quad \mu \text{ is the mean, and denoted by } X \sim \text{Exponential}(\lambda).
\]
\end{eulerformula}
\begin{eulerprompt}
>plot2d(randexponential(1,1000,2),>distribution):
\end{eulerprompt}
\eulerimg{17}{images/EMT4Statistika_Wahyu Rananda Westri_22305144039_Matematika B-015.png}
\begin{eulercomment}
Catatan tambahan :\\
randexponential : matrix acak dari distribusi eksponensial

Untuk banyak distribusi, Euler dapat menghitung fungsi distribusi dan
inversenya.
\end{eulercomment}
\begin{eulerprompt}
>plot2d("normaldis",-4,4): 
\end{eulerprompt}
\eulerimg{17}{images/EMT4Statistika_Wahyu Rananda Westri_22305144039_Matematika B-016.png}
\begin{eulercomment}
Berikut adalah salah satu cara untuk memplot kuantil.
\end{eulercomment}
\begin{eulerprompt}
>plot2d("qnormal(x,1,1.5)",-4,6);  ...
>plot2d("qnormal(x,1,1.5)",a=2,b=5,>add,>filled):
\end{eulerprompt}
\eulerimg{17}{images/EMT4Statistika_Wahyu Rananda Westri_22305144039_Matematika B-017.png}
\begin{eulerformula}
\[
\text{normaldis(x,m,d)}=\int_{-\infty}^x \frac{1}{d\sqrt{2\pi}}e^{-\frac{1}{2}(\frac{t-m}{d})^2}\ dt.
\]
\end{eulerformula}
\begin{eulercomment}
Probabilitas berada di area hijau adalah sebagai berikut.
\end{eulercomment}
\begin{eulerprompt}
>normaldis(5,1,1.5)-normaldis(2,1,1.5)
\end{eulerprompt}
\begin{euleroutput}
  0.248662156979
\end{euleroutput}
\begin{eulercomment}
Ini dapat dihitung secara numerik dengan integral berikut.\\
\end{eulercomment}
\begin{eulerformula}
\[
\int_2^5 \frac{1}{1.5\sqrt{2\pi}}e^{-\frac{1}{2}(\frac{x-1}{1.5})^2}\ dx.
\]
\end{eulerformula}
\begin{eulerprompt}
>gauss("qnormal(x,1,1.5)",2,5)
\end{eulerprompt}
\begin{euleroutput}
  0.248662156979
\end{euleroutput}
\begin{eulercomment}
Mari bandingkan distribusi binomial dengan distribusi normal dengan
rata-rata dan deviasi standar yang sama. Fungsi invbindis()
menyelesaikan interpolasi linear antara nilai-nilai bulat.
\end{eulercomment}
\begin{eulerprompt}
>invbindis(0.95,1000,0.5), invnormaldis(0.95,500,0.5*sqrt(1000))
\end{eulerprompt}
\begin{euleroutput}
  525.516721219
  526.007419394
\end{euleroutput}
\begin{eulercomment}
Fungsi qdis() adalah fungsi densitas distribusi chi-square. Seperti
biasa, Euler memetakan vektor ke fungsi ini. Dengan demikian, kita
dapat dengan mudah membuat plot untuk semua distribusi chi-square
dengan derajat 5 hingga 30 dengan cara berikut.
\end{eulercomment}
\begin{eulerprompt}
>plot2d("qchidis(x,(5:5:50)')",0,50):
\end{eulerprompt}
\eulerimg{17}{images/EMT4Statistika_Wahyu Rananda Westri_22305144039_Matematika B-019.png}
\begin{eulercomment}
Euler memiliki fungsi yang akurat untuk mengevaluasi distribusi. Mari
kita periksa chidis() dengan sebuah integral.

Penamaan fungsi tersebut mencoba konsisten. Misalnya,

- distribusi chi-square adalah chidis(),\\
- fungsi inversenya adalah invchidis(),\\
- fungsi densitasnya adalah qchidis().

Komplemen dari distribusi (ekor atas) disebut chicdis().
\end{eulercomment}
\begin{eulerprompt}
>chidis(1.5,2), integrate("qchidis(x,2)",0,1.5)
\end{eulerprompt}
\begin{euleroutput}
  0.527633447259
  0.527633447259
\end{euleroutput}
\eulerheading{Distribusi Diskret}
\begin{eulercomment}
Untuk mendefinisikan distribusi diskret Anda sendiri, Anda dapat
menggunakan metode berikut.

Pertama, kita atur fungsi distribusi.
\end{eulercomment}
\begin{eulerprompt}
>wd = 0|((1:6)+[-0.01,0.01,0,0,0,0])/6
\end{eulerprompt}
\begin{euleroutput}
  [0,  0.165,  0.335,  0.5,  0.666667,  0.833333,  1]
\end{euleroutput}
\begin{eulercomment}
Artinya, dengan probabilitas wd[i+1] - wd[i], kita menghasilkan nilai
acak i.

Ini hampir merupakan distribusi seragam. Mari kita tentukan pembangkit
angka acak untuk ini. Fungsi find(v, x) mencari nilai x dalam vektor
v. Fungsi ini juga berfungsi untuk vektor x.
\end{eulercomment}
\begin{eulerprompt}
>function wrongdice (n,m) := find(wd,random(n,m))
\end{eulerprompt}
\begin{eulercomment}
Kesalahan tersebut sangat halus sehingga kita hanya bisa melihatnya
dengan sangat banyak iterasi.
\end{eulercomment}
\begin{eulerprompt}
>columnsplot(getmultiplicities(1:6,wrongdice(1,1000000))):
\end{eulerprompt}
\eulerimg{17}{images/EMT4Statistika_Wahyu Rananda Westri_22305144039_Matematika B-020.png}
\begin{eulercomment}
Berikut adalah fungsi sederhana untuk memeriksa distribusi seragam
dari nilai-nilai 1 hingga K dalam vektor v. Kita menerima hasilnya
jika untuk semua frekuensi

\end{eulercomment}
\begin{eulerformula}
\[
\left|f_i-\frac{1}{K}\right| < \frac{\delta}{\sqrt{n}}.
\]
\end{eulerformula}
\begin{eulerprompt}
>function checkrandom (v, delta=1) ...
\end{eulerprompt}
\begin{eulerudf}
    K=max(v); n=cols(v);
    fr=getfrequencies(v,1:K);
    return max(fr/n-1/K)<delta/sqrt(n);
    endfunction
\end{eulerudf}
\begin{eulercomment}
Memang, fungsi tersebut menolak distribusi seragam.
\end{eulercomment}
\begin{eulerprompt}
>checkrandom(wrongdice(1,1000000))
\end{eulerprompt}
\begin{euleroutput}
  0
\end{euleroutput}
\begin{eulercomment}
Dan itu menerima pembangkit angka acak bawaan.
\end{eulercomment}
\begin{eulerprompt}
>checkrandom(intrandom(1,1000000,6))
\end{eulerprompt}
\begin{euleroutput}
  1
\end{euleroutput}
\begin{eulercomment}
Kita dapat menghitung distribusi binomial. Pertama, ada fungsi
binomialsum(), yang mengembalikan probabilitas i atau kurang hasil
dalam n percobaan.
\end{eulercomment}
\begin{eulerprompt}
>bindis(410,1000,0.4)//Distribusi Binomial Kumulatif
\end{eulerprompt}
\begin{euleroutput}
  0.751401349654
\end{euleroutput}
\begin{eulercomment}
Fungsi invers Beta digunakan untuk menghitung interval kepercayaan
Clopper-Pearson untuk parameter p. Tingkat kepercayaan default adalah
alpha.

Makna dari interval ini adalah bahwa jika p berada di luar interval
tersebut, hasil yang diamati 410 dari 1000 adalah hal yang jarang
terjadi.
\end{eulercomment}
\begin{eulerprompt}
>clopperpearson(410,1000)
\end{eulerprompt}
\begin{euleroutput}
  [0.37932,  0.441212]
\end{euleroutput}
\begin{eulercomment}
Perintah-perintah berikut adalah cara langsung untuk mendapatkan hasil
di atas. Namun, untuk nilai n yang besar, penjumlahan langsung tidak
akurat dan lambat.
\end{eulercomment}
\begin{eulerprompt}
>p=0.4; i=0:410; n=1000; sum(bin(n,i)*p^i*(1-p)^(n-i))
\end{eulerprompt}
\begin{euleroutput}
  0.751401349655
\end{euleroutput}
\begin{eulercomment}
Sekadar informasi, invbinsum() menghitung invers dari binomialsum().
\end{eulercomment}
\begin{eulerprompt}
>invbindis(0.75,1000,0.4)
\end{eulerprompt}
\begin{euleroutput}
  409.932733047
\end{euleroutput}
\begin{eulercomment}
Dalam permainan Bridge, kita asumsikan ada 5 kartu istimewa (dari
total 52 kartu) dalam dua tangan (26 kartu). Mari kita hitung
probabilitas dari distribusi yang lebih buruk daripada 3:2 (misalnya
0:5, 1:4, 4:1, atau 5:0).
\end{eulercomment}
\begin{eulerprompt}
>2*hypergeomsum(1,5,13,26)
\end{eulerprompt}
\begin{euleroutput}
  0.321739130435
\end{euleroutput}
\begin{eulercomment}
Ada juga simulasi distribusi multinomial.
\end{eulercomment}
\begin{eulerprompt}
>randmultinomial(10,1000,[0.4,0.1,0.5])
\end{eulerprompt}
\begin{euleroutput}
            418            92           490 
            445            90           465 
            403           114           483 
            405            95           500 
            384           107           509 
            414            93           493 
            419            90           491 
            394           101           505 
            382           103           515 
            403           128           469 
\end{euleroutput}
\eulerheading{Plotting Data}
\begin{eulercomment}
Untuk memplot data, kita mencoba hasil pemilihan umum Jerman sejak
tahun 1990, diukur dalam kursi.
\end{eulercomment}
\begin{eulerprompt}
>BW := [ ...
>1990,662,319,239,79,8,17; ...
>1994,672,294,252,47,49,30; ...
>1998,669,245,298,43,47,36; ...
>2002,603,248,251,47,55,2; ...
>2005,614,226,222,61,51,54; ...
>2009,622,239,146,93,68,76; ...
>2013,631,311,193,0,63,64];
\end{eulerprompt}
\begin{eulercomment}
Untuk partai-partai politik, kita menggunakan string nama-nama partai.
\end{eulercomment}
\begin{eulerprompt}
>P:=["CDU/CSU","SPD","FDP","Gr","Li"];
\end{eulerprompt}
\begin{eulercomment}
Mari mencetak persentasenya dengan rapi.

Pertama, kita ekstrak kolom-kolom yang diperlukan. Kolom 3 hingga 7
adalah kursi-kursi masing-masing partai, dan kolom 2 adalah total
jumlah kursi. Kolom adalah tahun pemilihan umum.
\end{eulercomment}
\begin{eulerprompt}
>BT:=BW[,3:7]; BT:=BT/sum(BT); YT:=BW[,1]';
\end{eulerprompt}
\begin{eulercomment}
Kemudian kita cetak statistiknya dalam bentuk tabel. Kita gunakan
nama-nama partai sebagai judul kolom, dan tahun-tahun sebagai judul
baris. Lebar default untuk kolom-kolom adalah wc=10, namun kita lebih
memilih output yang lebih padat. Kolom-kolom akan diperluas untuk
label-label kolom, jika diperlukan.
\end{eulercomment}
\begin{eulerprompt}
>writetable(BT*100,wc=6,dc=0,>fixed,labc=P,labr=YT)
\end{eulerprompt}
\begin{euleroutput}
         CDU/CSU   SPD   FDP    Gr    Li
    1990      48    36    12     1     3
    1994      44    38     7     7     4
    1998      37    45     6     7     5
    2002      41    42     8     9     0
    2005      37    36    10     8     9
    2009      38    23    15    11    12
    2013      49    31     0    10    10
\end{euleroutput}
\begin{eulercomment}
Perkalian matriks berikut mengekstrak jumlah persentase dari dua
partai besar, menunjukkan bahwa partai-partai kecil telah mendapatkan
perolehan kursi di parlemen hingga tahun 2009.
\end{eulercomment}
\begin{eulerprompt}
>BT1:=(BT.[1;1;0;0;0])'*100
\end{eulerprompt}
\begin{euleroutput}
  [84.29,  81.25,  81.1659,  82.7529,  72.9642,  61.8971,  79.8732]
\end{euleroutput}
\begin{eulercomment}
Ada juga plot statistik sederhana. Kita menggunakannya untuk
menampilkan garis dan titik secara bersamaan. Alternatifnya adalah
dengan memanggil plot2d dua kali dengan \textgreater{}add.
\end{eulercomment}
\begin{eulerprompt}
>statplot(YT,BT1,"b"):
\end{eulerprompt}
\eulerimg{17}{images/EMT4Statistika_Wahyu Rananda Westri_22305144039_Matematika B-022.png}
\begin{eulercomment}
Tentukan beberapa warna untuk setiap partai.
\end{eulercomment}
\begin{eulerprompt}
>CP:=[rgb(0.5,0.5,0.5),red,yellow,green,rgb(0.8,0,0)];
\end{eulerprompt}
\begin{eulercomment}
Sekarang kita dapat memplot hasil pemilihan tahun 2009 dan perubahan
hasilnya dalam satu plot menggunakan fungsi figure. Kita dapat
menambahkan vektor kolom ke setiap plot.
\end{eulercomment}
\begin{eulerprompt}
>figure(2,1);  ...
>figure(1); columnsplot(BW[6,3:7],P,color=CP); ...
>figure(2); columnsplot(BW[6,3:7]-BW[5,3:7],P,color=CP);  ...
>figure(0):
\end{eulerprompt}
\eulerimg{17}{images/EMT4Statistika_Wahyu Rananda Westri_22305144039_Matematika B-023.png}
\begin{eulercomment}
Plot data menggabungkan baris-baris data statistik dalam satu plot.
\end{eulercomment}
\begin{eulerprompt}
>J:=BW[,1]'; DP:=BW[,3:7]'; ...
>dataplot(YT,BT',color=CP);  ...
>labelbox(P,colors=CP,styles="[]",>points,w=0.2,x=0.3,y=0.4):
\end{eulerprompt}
\eulerimg{17}{images/EMT4Statistika_Wahyu Rananda Westri_22305144039_Matematika B-024.png}
\begin{eulercomment}
Plot kolom 3D menunjukkan baris-baris data statistik dalam bentuk
kolom. Kita menyediakan label untuk baris dan kolom. Sudut (angle)
adalah sudut pandang tampilan.
\end{eulercomment}
\begin{eulerprompt}
>columnsplot3d(BT,scols=P,srows=YT, ...
>  angle=30°,ccols=CP):
\end{eulerprompt}
\eulerimg{17}{images/EMT4Statistika_Wahyu Rananda Westri_22305144039_Matematika B-025.png}
\begin{eulercomment}
Representasi lainnya adalah plot mozaik. Perlu diingat bahwa
kolom-kolom plot ini merepresentasikan kolom-kolom dari matriks di
sini. Karena panjang label CDU/CSU, kita menggunakan jendela yang
lebih kecil dari biasanya.
\end{eulercomment}
\begin{eulerprompt}
>shrinkwindow(>smaller);  ...
>mosaicplot(BT',srows=YT,scols=P,color=CP,style="#"); ...
>shrinkwindow():
\end{eulerprompt}
\eulerimg{17}{images/EMT4Statistika_Wahyu Rananda Westri_22305144039_Matematika B-026.png}
\begin{eulercomment}
Kita juga dapat membuat diagram lingkaran (pie chart). Karena hitam
dan kuning membentuk koalisi, kita akan menyusun ulang elemen-elemen
tersebut.
\end{eulercomment}
\begin{eulerprompt}
>i=[1,3,5,4,2]; piechart(BW[6,3:7][i],color=CP[i],lab=P[i]):
\end{eulerprompt}
\eulerimg{17}{images/EMT4Statistika_Wahyu Rananda Westri_22305144039_Matematika B-027.png}
\begin{eulercomment}
Di sini adalah jenis plot yang lain.
\end{eulercomment}
\begin{eulerprompt}
>starplot(normal(1,10)+4,lab=1:10,>rays):
\end{eulerprompt}
\eulerimg{17}{images/EMT4Statistika_Wahyu Rananda Westri_22305144039_Matematika B-028.png}
\begin{eulercomment}
Beberapa plot dalam plot2d cocok untuk data statis. Berikut adalah
plot impuls dari data acak yang terdistribusi secara merata dalam
[0,1].
\end{eulercomment}
\begin{eulerprompt}
>plot2d(makeimpulse(1:10,random(1,10)),>bar):
\end{eulerprompt}
\eulerimg{17}{images/EMT4Statistika_Wahyu Rananda Westri_22305144039_Matematika B-029.png}
\begin{eulercomment}
But for exponentially distributed data, we may need a logarithmic plot.
\end{eulercomment}
\begin{eulerprompt}
>logimpulseplot(1:10,-log(random(1,10))*10):
\end{eulerprompt}
\eulerimg{17}{images/EMT4Statistika_Wahyu Rananda Westri_22305144039_Matematika B-030.png}
\begin{eulercomment}
Fungsi `columnsplot()` lebih mudah digunakan, karena hanya memerlukan
vektor nilai. Selain itu, kita dapat mengatur label sesuai keinginan
kita, seperti yang telah kami tunjukkan dalam tutorial ini sebelumnya.

Berikut adalah aplikasi lain, di mana kita menghitung karakter dalam
sebuah kalimat dan membuat plot statistiknya.
\end{eulercomment}
\begin{eulerprompt}
>v=strtochar("the quick brown fox jumps over the lazy dog"); ...
>w=ascii("a"):ascii("z"); x=getmultiplicities(w,v); ...
>cw=[]; for k=w; cw=cw|char(k); end; ...
>columnsplot(x,lab=cw,width=0.05):
\end{eulerprompt}
\eulerimg{17}{images/EMT4Statistika_Wahyu Rananda Westri_22305144039_Matematika B-031.png}
\begin{eulercomment}
Juga mungkin untuk secara manual mengatur sumbu-sumbu.
\end{eulercomment}
\begin{eulerprompt}
>n=10; p=0.4; i=0:n; x=bin(n,i)*p^i*(1-p)^(n-i); ...
>columnsplot(x,lab=i,width=0.05,<frame,<grid); ...
>yaxis(0,0:0.1:1,style="->",>left); xaxis(0,style="."); ...
>label("p",0,0.25), label("i",11,0); ...
>textbox(["Binomial distribution","with p=0.4"]):
\end{eulerprompt}
\eulerimg{17}{images/EMT4Statistika_Wahyu Rananda Westri_22305144039_Matematika B-032.png}
\begin{eulercomment}
Berikut adalah cara untuk membuat plot frekuensi angka dalam sebuah
vektor.

Kita membuat vektor dari angka-angka acak integer dari 1 hingga 6.
\end{eulercomment}
\begin{eulerprompt}
>v:=intrandom(1,10,10)
\end{eulerprompt}
\begin{euleroutput}
  [4,  5,  2,  6,  1,  10,  8,  4,  1,  2]
\end{euleroutput}
\begin{eulercomment}
Kemudian ekstrak angka-angka unik dalam vektor v.
\end{eulercomment}
\begin{eulerprompt}
>vu:=unique(v)
\end{eulerprompt}
\begin{euleroutput}
  [1,  2,  4,  5,  6,  8,  10]
\end{euleroutput}
\begin{eulercomment}
Dan plot frekuensi tersebut dalam sebuah plot kolom.
\end{eulercomment}
\begin{eulerprompt}
>columnsplot(getmultiplicities(vu,v),lab=vu,style="/"):
\end{eulerprompt}
\eulerimg{17}{images/EMT4Statistika_Wahyu Rananda Westri_22305144039_Matematika B-033.png}
\begin{eulercomment}
Kami ingin mendemonstrasikan fungsi-fungsi untuk distribusi empiris
dari nilai-nilai.
\end{eulercomment}
\begin{eulerprompt}
>x=normal(1,20);
\end{eulerprompt}
\begin{eulercomment}
Fungsi `empdist(x, vs)` memerlukan sebuah larik nilai yang telah
diurutkan. Oleh karena itu, kita perlu mengurutkan x sebelum kita
dapat menggunakannya.
\end{eulercomment}
\begin{eulerprompt}
>xs=sort(x);
\end{eulerprompt}
\begin{eulercomment}
Kemudian kita membuat plot distribusi empiris dan beberapa batang
kepadatan dalam satu plot. Kali ini, daripada menggunakan plot batang
untuk distribusi, kita akan menggunakan plot gigi gergaji (sawtooth
plot).
\end{eulercomment}
\begin{eulerprompt}
>figure(2,1); ...
>figure(1); plot2d("empdist",-4,4;xs); ...
>figure(2); plot2d(histo(x,v=-4:0.2:4,<bar));  ...
>figure(0):
\end{eulerprompt}
\eulerimg{17}{images/EMT4Statistika_Wahyu Rananda Westri_22305144039_Matematika B-034.png}
\begin{eulercomment}
Diagram hamburan (scatter plot) mudah dilakukan dalam Euler dengan
plot titik biasa. Grafik berikut menunjukkan bahwa X dan X+Y jelas
berkorelasi positif.
\end{eulercomment}
\begin{eulerprompt}
>x=normal(1,100); plot2d(x,x+rotright(x),>points,style=".."):
\end{eulerprompt}
\eulerimg{17}{images/EMT4Statistika_Wahyu Rananda Westri_22305144039_Matematika B-035.png}
\begin{eulercomment}
Seringkali, kita ingin membandingkan dua sampel dengan distribusi yang
berbeda. Hal ini dapat dilakukan dengan menggunakan plot
kuantil-kuantil (quantile-quantile plot).

Untuk sebuah uji, kita mencoba distribusi t-student dan distribusi
eksponensial.
\end{eulercomment}
\begin{eulerprompt}
>x=randt(1,1000,5); y=randnormal(1,1000,mean(x),dev(x)); ...
>plot2d("x",r=6,style="--",yl="normal",xl="student-t",>vertical); ...
>plot2d(sort(x),sort(y),>points,color=red,style="x",>add):
\end{eulerprompt}
\eulerimg{17}{images/EMT4Statistika_Wahyu Rananda Westri_22305144039_Matematika B-036.png}
\begin{eulercomment}
Grafik tersebut dengan jelas menunjukkan bahwa nilai yang
terdistribusi secara normal cenderung lebih kecil di ujung-ujung
ekstrem.

Jika kita memiliki dua distribusi dengan ukuran yang berbeda, kita
dapat memperbesar yang lebih kecil atau mengecilkan yang lebih besar.
Fungsi berikut baik digunakan untuk keduanya. Ini mengambil nilai
median dengan persentase antara 0 dan 1.
\end{eulercomment}
\begin{eulerprompt}
>function medianexpand (x,n) := median(x,p=linspace(0,1,n-1));
\end{eulerprompt}
\begin{eulercomment}
Mari kita bandingkan dua distribusi yang sama.
\end{eulercomment}
\begin{eulerprompt}
>x=random(1000); y=random(400); ...
>plot2d("x",0,1,style="--"); ...
>plot2d(sort(medianexpand(x,400)),sort(y),>points,color=red,style="x",>add):
\end{eulerprompt}
\eulerimg{17}{images/EMT4Statistika_Wahyu Rananda Westri_22305144039_Matematika B-037.png}
\eulerheading{Regresi dan Korelasi}
\begin{eulercomment}
Regresi linear dapat dilakukan dengan menggunakan fungsi polyfit()
atau berbagai fungsi fit lainnya.

Untuk memulai, kita dapat menemukan garis regresi untuk data univariat
dengan polyfit(x, y, 1).
\end{eulercomment}
\begin{eulerprompt}
>x=1:10; y=[2,3,1,5,6,3,7,8,9,8]; writetable(x'|y',labc=["x","y"])
\end{eulerprompt}
\begin{euleroutput}
           x         y
           1         2
           2         3
           3         1
           4         5
           5         6
           6         3
           7         7
           8         8
           9         9
          10         8
\end{euleroutput}
\begin{eulercomment}
Kita ingin membandingkan hasil regresi tanpa bobot (non-weighted) dan
dengan bobot (weighted). Pertama-tama, mari lihat koefisien regresi
linearnya.
\end{eulercomment}
\begin{eulerprompt}
>p=polyfit(x,y,1)
\end{eulerprompt}
\begin{euleroutput}
  [0.733333,  0.812121]
\end{euleroutput}
\begin{eulercomment}
Sekarang mari lihat koefisien dengan bobot yang menekankan nilai-nilai
terakhir.
\end{eulercomment}
\begin{eulerprompt}
>w &= "exp(-(x-10)^2/10)"; pw=polyfit(x,y,1,w=w(x))
\end{eulerprompt}
\begin{euleroutput}
  [4.71566,  0.38319]
\end{euleroutput}
\begin{eulercomment}
Kita gabungkan semuanya dalam satu plot untuk titik-titik data, garis
regresi, dan untuk bobot yang digunakan.
\end{eulercomment}
\begin{eulerprompt}
>figure(2,1);  ...
>figure(1); statplot(x,y,"b",xl="Regression"); ...
>  plot2d("evalpoly(x,p)",>add,color=blue,style="--"); ...
>  plot2d("evalpoly(x,pw)",5,10,>add,color=red,style="--"); ...
>figure(2); plot2d(w,1,10,>filled,style="/",fillcolor=red,xl=w); ...
>figure(0):
\end{eulerprompt}
\eulerimg{17}{images/EMT4Statistika_Wahyu Rananda Westri_22305144039_Matematika B-038.png}
\begin{eulercomment}
Untuk contoh lain, kita membaca hasil survei tentang mahasiswa, usia
mereka, usia orang tua mereka, dan jumlah saudara kandung dari sebuah
file.

Tabel ini berisi "m" dan "f" dalam kolom kedua. Kita menggunakan
variabel tok2 untuk mengatur terjemahan yang sesuai daripada
membiarkan readtable() mengumpulkan terjemahan.
\end{eulercomment}
\begin{eulerprompt}
>\{MS,hd\}:=readtable("table1.dat",tok2:=["m","f"]);  ...
>writetable(MS,labc=hd,tok2:=["m","f"]);
\end{eulerprompt}
\begin{euleroutput}
  Could not open the file
  table1.dat
  for reading!
  Try "trace errors" to inspect local variables after errors.
  readtable:
      if filename!=none then open(filename,"r"); endif;
\end{euleroutput}
\begin{eulercomment}
Bagaimana usia-usia ini bergantung satu sama lain? Kesimpulan awal
dapat diperoleh dari scatterplot pasangan (pairwise scatterplot).
\end{eulercomment}
\begin{eulerprompt}
>scatterplots(tablecol(MS,3:5),hd[3:5]):
\end{eulerprompt}
\begin{euleroutput}
  Variable or function MS not found.
  Error in:
  scatterplots(tablecol(MS,3:5),hd[3:5]): ...
                          ^
\end{euleroutput}
\begin{eulercomment}
Jelas bahwa usia ayah dan ibu saling bergantung. Mari kita tentukan
dan plot garis regresinya.
\end{eulercomment}
\begin{eulerprompt}
>cs:=MS[,4:5]'; ps:=polyfit(cs[1],cs[2],1)
\end{eulerprompt}
\begin{euleroutput}
  MS is not a variable!
  Error in:
  cs:=MS[,4:5]'; ps:=polyfit(cs[1],cs[2],1) ...
              ^
\end{euleroutput}
\begin{eulercomment}
Ini jelas bukan model yang tepat. Garis regresinya seharusnya adalah s
= 17 + 0.74t, di mana t adalah usia ibu dan s adalah usia ayah.
Perbedaan usia mungkin sedikit bergantung pada usia, tetapi tidak
sebanyak itu.

Sebaliknya, kami mencurigai bahwa fungsi seperti s = a + t. Kemudian,
a adalah rata-rata dari s-t. Itu adalah perbedaan usia rata-rata
antara ayah dan ibu.
\end{eulercomment}
\begin{eulerprompt}
>da:=mean(cs[2]-cs[1])
\end{eulerprompt}
\begin{euleroutput}
  cs is not a variable!
  Error in:
  da:=mean(cs[2]-cs[1]) ...
                ^
\end{euleroutput}
\begin{eulercomment}
Mari kita plot ini dalam satu scatter plot.
\end{eulercomment}
\begin{eulerprompt}
>plot2d(cs[1],cs[2],>points);  ...
>plot2d("evalpoly(x,ps)",color=red,style=".",>add);  ...
>plot2d("x+da",color=blue,>add):
\end{eulerprompt}
\begin{euleroutput}
  cs is not a variable!
  Error in:
  plot2d(cs[1],cs[2],>points);  plot2d("evalpoly(x,ps)",color=re ...
              ^
\end{euleroutput}
\begin{eulercomment}
Berikut adalah box plot dari dua usia. Ini hanya menunjukkan bahwa
usia-usia tersebut berbeda.
\end{eulercomment}
\begin{eulerprompt}
>boxplot(cs,["mothers","fathers"]):
\end{eulerprompt}
\begin{euleroutput}
  Variable or function cs not found.
  Error in:
  boxplot(cs,["mothers","fathers"]): ...
            ^
\end{euleroutput}
\begin{eulercomment}
Menarik bahwa perbedaan median tidak sebesar perbedaan mean.
\end{eulercomment}
\begin{eulerprompt}
>median(cs[2])-median(cs[1])
\end{eulerprompt}
\begin{euleroutput}
  cs is not a variable!
  Error in:
  median(cs[2])-median(cs[1]) ...
              ^
\end{euleroutput}
\begin{eulercomment}
Koefisien korelasi menunjukkan adanya korelasi positif.
\end{eulercomment}
\begin{eulerprompt}
>correl(cs[1],cs[2])
\end{eulerprompt}
\begin{euleroutput}
  cs is not a variable!
  Error in:
  correl(cs[1],cs[2]) ...
              ^
\end{euleroutput}
\begin{eulercomment}
Korelasi peringkat adalah ukuran untuk urutan yang sama dalam kedua
vektor. Juga sangat positif.
\end{eulercomment}
\begin{eulerprompt}
>rankcorrel(cs[1],cs[2])
\end{eulerprompt}
\begin{euleroutput}
  cs is not a variable!
  Error in:
  rankcorrel(cs[1],cs[2]) ...
                  ^
\end{euleroutput}
\eulerheading{Membuat Fungsi Baru}
\begin{eulercomment}
Tentu saja, bahasa EMT dapat digunakan untuk membuat fungsi-fungsi
baru. Misalnya, kita dapat mendefinisikan fungsi kesarjanaan
(skewness).

\end{eulercomment}
\begin{eulerformula}
\[
\text{sk}(x) = \dfrac{\sqrt{n} \sum_i (x_i-m)^3}{\left(\sum_i (x_i-m)^2\right)^{3/2}}
\]
\end{eulerformula}
\begin{eulercomment}
di mana m adalah rata-rata dari x.
\end{eulercomment}
\begin{eulerprompt}
>function skew (x:vector) ...
\end{eulerprompt}
\begin{eulerudf}
  m=mean(x);
  return sqrt(cols(x))*sum((x-m)^3)/(sum((x-m)^2))^(3/2);
  endfunction
\end{eulerudf}
\begin{eulercomment}
Seperti yang Anda lihat, kita dengan mudah dapat menggunakan bahasa
matriks untuk mendapatkan implementasi yang sangat singkat dan
efisien. Mari kita coba fungsi ini.
\end{eulercomment}
\begin{eulerprompt}
>data=normal(20); skew(normal(10))
\end{eulerprompt}
\begin{euleroutput}
  -0.0794571577159
\end{euleroutput}
\begin{eulercomment}
Berikut adalah fungsi lain yang disebut koefisien skewness Pearson.
\end{eulercomment}
\begin{eulerprompt}
>function skew1 (x) := 3*(mean(x)-median(x))/dev(x)
>skew1(data)
\end{eulerprompt}
\begin{euleroutput}
  -1.04418407161
\end{euleroutput}
\eulerheading{Simulasi Monte Carlo}
\begin{eulercomment}
Euler dapat digunakan untuk mensimulasikan peristiwa acak. Kami telah
melihat contoh-contoh sederhana di atas. Berikut adalah contoh lain
yang mensimulasikan 1000 kali lemparan tiga dadu, dan menghitung
distribusi jumlahnya.
\end{eulercomment}
\begin{eulerprompt}
>ds:=sum(intrandom(1000,3,6))';  fs=getmultiplicities(3:18,ds)
\end{eulerprompt}
\begin{euleroutput}
  [6,  22,  29,  43,  65,  92,  125,  132,  115,  118,  92,  75,  45,
  31,  8,  2]
\end{euleroutput}
\begin{eulercomment}
Sekarang kita dapat membuat plot hasil simulasi ini.
\end{eulercomment}
\begin{eulerprompt}
>columnsplot(fs,lab=3:18):
\end{eulerprompt}
\eulerimg{17}{images/EMT4Statistika_Wahyu Rananda Westri_22305144039_Matematika B-040.png}
\begin{eulercomment}
Untuk menentukan distribusi yang diharapkan tidaklah mudah. Kita
menggunakan rekursi tingkat lanjut untuk ini.

Fungsi berikut menghitung jumlah cara di mana angka k dapat
direpresentasikan sebagai jumlah dari n angka dalam rentang 1 hingga
m. Ini bekerja secara rekursif dengan cara yang jelas.
\end{eulercomment}
\begin{eulerprompt}
>function map countways (k; n, m) ...
\end{eulerprompt}
\begin{eulerudf}
    if n==1 then return k>=1 && k<=m
    else
      sum=0; 
      loop 1 to m; sum=sum+countways(k-#,n-1,m); end;
      return sum;
    end;
  endfunction
\end{eulerudf}
\begin{eulercomment}
Berikut adalah hasilnya untuk tiga lemparan dadu.
\end{eulercomment}
\begin{eulerprompt}
>countways(5:25,5,5)
\end{eulerprompt}
\begin{euleroutput}
  [1,  5,  15,  35,  70,  121,  185,  255,  320,  365,  381,  365,  320,
  255,  185,  121,  70,  35,  15,  5,  1]
\end{euleroutput}
\begin{eulerprompt}
>cw=countways(3:18,3,6)
\end{eulerprompt}
\begin{euleroutput}
  [1,  3,  6,  10,  15,  21,  25,  27,  27,  25,  21,  15,  10,  6,  3,
  1]
\end{euleroutput}
\begin{eulercomment}
Kita tambahkan nilai-nilai yang diharapkan ke dalam plot.
\end{eulercomment}
\begin{eulerprompt}
>plot2d(cw/6^3*1000,>add); plot2d(cw/6^3*1000,>points,>add):
\end{eulerprompt}
\eulerimg{17}{images/EMT4Statistika_Wahyu Rananda Westri_22305144039_Matematika B-041.png}
\begin{eulercomment}
Untuk simulasi lainnya, deviasi nilai rata-rata dari n variabel acak
yang terdistribusi normal antara 0 dan 1 adalah 1/sqrt(n).
\end{eulercomment}
\begin{eulerprompt}
>longformat; 1/sqrt(10)
\end{eulerprompt}
\begin{euleroutput}
  0.316227766017
\end{euleroutput}
\begin{eulercomment}
Mari kita periksa ini dengan simulasi. Kami menghasilkan 10.000 kali
10 vektor acak.
\end{eulercomment}
\begin{eulerprompt}
>M=normal(10000,10); dev(mean(M)')
\end{eulerprompt}
\begin{euleroutput}
  0.314944511075
\end{euleroutput}
\begin{eulerprompt}
>plot2d(mean(M)',>distribution):
\end{eulerprompt}
\eulerimg{17}{images/EMT4Statistika_Wahyu Rananda Westri_22305144039_Matematika B-042.png}
\begin{eulercomment}
Median dari 10 angka acak yang terdistribusi normal antara 0 dan 1
memiliki deviasi yang lebih besar.
\end{eulercomment}
\begin{eulerprompt}
>dev(median(M)')
\end{eulerprompt}
\begin{euleroutput}
  0.370541708221
\end{euleroutput}
\begin{eulercomment}
Karena kita dapat dengan mudah menghasilkan perjalanan acak (random
walks), kita dapat mensimulasikan proses Wiener. Kita ambil 1000
langkah dari 1000 proses ini. Kemudian kita membuat plot dari
simpangan baku (standard deviation) dan rata-rata langkah ke-n dari
proses-proses ini bersama dengan nilai-nilai yang diharapkan dalam
warna merah.
\end{eulercomment}
\begin{eulerprompt}
>n=1000; m=1000; M=cumsum(normal(n,m)/sqrt(m)); ...
>t=(1:n)/n; figure(2,1); ...
>figure(1); plot2d(t,mean(M')'); plot2d(t,0,color=red,>add); ...
>figure(2); plot2d(t,dev(M')'); plot2d(t,sqrt(t),color=red,>add); ...
>figure(0):
\end{eulerprompt}
\eulerimg{17}{images/EMT4Statistika_Wahyu Rananda Westri_22305144039_Matematika B-043.png}
\eulerheading{Uji Statistik}
\begin{eulercomment}
Uji statistik merupakan alat penting dalam statistik. Dalam Euler,
banyak uji statistik telah diimplementasikan. Semua uji ini
mengembalikan kesalahan yang kita terima jika kita menolak hipotesis
nol.

Sebagai contoh, kita menguji lemparan dadu untuk distribusi seragam.
Pada 600 lemparan, kita mendapatkan nilai-nilai berikut, yang kita
masukkan ke dalam uji chi-kuadrat.
\end{eulercomment}
\begin{eulerprompt}
>chitest([90,103,114,101,103,89],dup(100,6)')
\end{eulerprompt}
\begin{euleroutput}
  0.498830517952
\end{euleroutput}
\begin{eulercomment}
Uji chi-kuadrat juga memiliki mode yang menggunakan simulasi Monte
Carlo untuk menguji statistik. Hasilnya seharusnya hampir sama.
Parameter \textgreater{}p menginterpretasikan vektor y sebagai vektor probabilitas.
\end{eulercomment}
\begin{eulerprompt}
>chitest([90,103,114,101,103,89],dup(1/6,6)',>p,>montecarlo)
\end{eulerprompt}
\begin{euleroutput}
  0.507
\end{euleroutput}
\begin{eulercomment}
Kesalahan ini jauh terlalu besar. Jadi kita tidak bisa menolak
distribusi seragam. Ini tidak membuktikan bahwa dadu kita adil. Tapi
kita tidak bisa menolak hipotesis kita.

Selanjutnya, kita menghasilkan 1000 lemparan dadu menggunakan
pembangkit bilangan acak, dan melakukan uji yang sama.
\end{eulercomment}
\begin{eulerprompt}
>n=1000; t=random([1,n*6]); chitest(count(t*6,6),dup(n,6)')
\end{eulerprompt}
\begin{euleroutput}
  0.247506458901
\end{euleroutput}
\begin{eulercomment}
Mari kita uji untuk nilai rata-rata 100 dengan uji t.
\end{eulercomment}
\begin{eulerprompt}
>s=200+normal([1,100])*10; ...
>ttest(mean(s),dev(s),100,200)
\end{eulerprompt}
\begin{euleroutput}
  0.261148883454
\end{euleroutput}
\begin{eulercomment}
Fungsi ttest() memerlukan nilai rata-rata, deviasi, jumlah data, dan
nilai rata-rata yang akan diuji.

Sekarang mari kita periksa dua pengukuran untuk nilai rata-rata yang
sama. Kita menolak hipotesis bahwa mereka memiliki nilai rata-rata
yang sama jika hasilnya \textless{}0,05.
\end{eulercomment}
\begin{eulerprompt}
>tcomparedata(normal(1,10),normal(1,10))
\end{eulerprompt}
\begin{euleroutput}
  0.116475902968
\end{euleroutput}
\begin{eulercomment}
Jika kita menambahkan bias pada salah satu distribusi, kita akan
mendapatkan lebih banyak penolakan. Ulangi simulasi ini beberapa kali
untuk melihat efeknya.
\end{eulercomment}
\begin{eulerprompt}
>tcomparedata(normal(1,10),normal(1,10)+2)
\end{eulerprompt}
\begin{euleroutput}
  5.770990267e-05
\end{euleroutput}
\begin{eulercomment}
Pada contoh berikut, kita menghasilkan 20 lemparan dadu acak 100 kali
dan menghitung jumlah angka satu (1) dalamnya. Rata-rata seharusnya
adalah 20/6 = 3,3 angka satu.
\end{eulercomment}
\begin{eulerprompt}
>R=random(100,20); R=sum(R*6<=1)'; mean(R)
\end{eulerprompt}
\begin{euleroutput}
  3.08
\end{euleroutput}
\begin{eulercomment}
Sekarang kita membandingkan jumlah angka satu dengan distribusi
binomial. Pertama-tama, kita membuat plot distribusi angka satu.
\end{eulercomment}
\begin{eulerprompt}
>plot2d(R,distribution=max(R)+1,even=1,style="\(\backslash\)/"):
\end{eulerprompt}
\eulerimg{17}{images/EMT4Statistika_Wahyu Rananda Westri_22305144039_Matematika B-044.png}
\begin{eulerprompt}
>t=count(R,21);
\end{eulerprompt}
\begin{eulercomment}
Kemudian kita menghitung nilai-nilai yang diharapkan.
\end{eulercomment}
\begin{eulerprompt}
>n=0:20; b=bin(20,n)*(1/6)^n*(5/6)^(20-n)*100;
\end{eulerprompt}
\begin{eulercomment}
Kita harus mengumpulkan beberapa angka untuk mendapatkan
kategori-kategori yang cukup besar.
\end{eulercomment}
\begin{eulerprompt}
>t1=sum(t[1:2])|t[3:7]|sum(t[8:21]); ...
>b1=sum(b[1:2])|b[3:7]|sum(b[8:21]);
\end{eulerprompt}
\begin{eulercomment}
Uji chi-kuadrat menolak hipotesis bahwa distribusi kita adalah
distribusi binomial, jika hasilnya \textless{}0,05.
\end{eulercomment}
\begin{eulerprompt}
>chitest(t1,b1)
\end{eulerprompt}
\begin{euleroutput}
  0.717403213286
\end{euleroutput}
\begin{eulercomment}
Contoh berikut berisi hasil dari dua kelompok orang (pria dan wanita,
misalnya) yang memilih salah satu dari enam partai.
\end{eulercomment}
\begin{eulerprompt}
>A=[23,37,43,52,64,74;27,39,41,49,63,76];  ...
>  writetable(A,wc=6,labr=["m","f"],labc=1:6)
\end{eulerprompt}
\begin{euleroutput}
             1     2     3     4     5     6
       m    23    37    43    52    64    74
       f    27    39    41    49    63    76
\end{euleroutput}
\begin{eulercomment}
Kita ingin menguji independensi suara dari jenis kelamin. Uji tabel
chi-kuadrat melakukannya. Hasilnya terlalu besar untuk menolak
independensi. Jadi, dari data ini, kita tidak bisa mengatakan apakah
pemilihan bergantung pada jenis kelamin atau tidak.
\end{eulercomment}
\begin{eulerprompt}
>tabletest(A)
\end{eulerprompt}
\begin{euleroutput}
  0.990701632326
\end{euleroutput}
\begin{eulercomment}
Berikut adalah tabel yang diharapkan jika kita mengasumsikan frekuensi
yang diamati dalam pemilihan.
\end{eulercomment}
\begin{eulerprompt}
>writetable(expectedtable(A),wc=6,dc=1,labr=["m","f"],labc=1:6)
\end{eulerprompt}
\begin{euleroutput}
             1     2     3     4     5     6
       m  24.9  37.9  41.9  50.3  63.3  74.7
       f  25.1  38.1  42.1  50.7  63.7  75.3
\end{euleroutput}
\begin{eulercomment}
Kita dapat menghitung koefisien kontingensi yang dikoreksi. Karena
nilai yang sangat mendekati 0, kita dapat menyimpulkan bahwa pemilihan
tidak bergantung pada jenis kelamin.
\end{eulercomment}
\begin{eulerprompt}
>contingency(A)
\end{eulerprompt}
\begin{euleroutput}
  0.0427225484717
\end{euleroutput}
\begin{eulercomment}
\begin{eulercomment}
\eulerheading{Beberapa Uji Lainnya}
\begin{eulercomment}
Selanjutnya, kita menggunakan analisis varians (uji F) untuk menguji
tiga sampel data yang terdistribusi normal untuk nilai rata-rata yang
sama. Metode ini disebut ANOVA (analisis varians). Dalam Euler,
digunakan fungsi varanalysis().
\end{eulercomment}
\begin{eulerprompt}
>x1=[109,111,98,119,91,118,109,99,115,109,94]; mean(x1),
\end{eulerprompt}
\begin{euleroutput}
  106.545454545
\end{euleroutput}
\begin{eulerprompt}
>x2=[120,124,115,139,114,110,113,120,117]; mean(x2),
\end{eulerprompt}
\begin{euleroutput}
  119.111111111
\end{euleroutput}
\begin{eulerprompt}
>x3=[120,112,115,110,105,134,105,130,121,111]; mean(x3)
\end{eulerprompt}
\begin{euleroutput}
  116.3
\end{euleroutput}
\begin{eulerprompt}
>varanalysis(x1,x2,x3)
\end{eulerprompt}
\begin{euleroutput}
  0.0138048221371
\end{euleroutput}
\begin{eulercomment}
Artinya, kita menolak hipotesis nilai rata-rata yang sama. Kita
melakukannya dengan probabilitas kesalahan sebesar 1,3\%.

Ada juga uji median, yang menolak sampel data dengan distribusi
rata-rata yang berbeda dengan menguji median dari sampel yang
digabungkan.
\end{eulercomment}
\begin{eulerprompt}
>a=[56,66,68,49,61,53,45,58,54];
>b=[72,81,51,73,69,78,59,67,65,71,68,71];
>mediantest(a,b)
\end{eulerprompt}
\begin{euleroutput}
  0.0241724220052
\end{euleroutput}
\begin{eulercomment}
Uji lain untuk kesetaraan adalah uji peringkat. Ini jauh lebih tajam
daripada uji median.
\end{eulercomment}
\begin{eulerprompt}
>ranktest(a,b)
\end{eulerprompt}
\begin{euleroutput}
  0.00199969612469
\end{euleroutput}
\begin{eulercomment}
Pada contoh berikut, kedua distribusi memiliki nilai rata-rata yang
sama.
\end{eulercomment}
\begin{eulerprompt}
>ranktest(random(1,100),random(1,50)*3-1)
\end{eulerprompt}
\begin{euleroutput}
  0.0908472946126
\end{euleroutput}
\begin{eulercomment}
Mari kita coba mensimulasikan dua perawatan a dan b yang diberikan
kepada orang-orang yang berbeda.
\end{eulercomment}
\begin{eulerprompt}
>a=[8.0,7.4,5.9,9.4,8.6,8.2,7.6,8.1,6.2,8.9];
>b=[6.8,7.1,6.8,8.3,7.9,7.2,7.4,6.8,6.8,8.1];
\end{eulerprompt}
\begin{eulercomment}
Uji signum (signum test) menentukan apakah a lebih baik daripada b.
\end{eulercomment}
\begin{eulerprompt}
>signtest(a,b)
\end{eulerprompt}
\begin{euleroutput}
  0.0546875
\end{euleroutput}
\begin{eulercomment}
Ini terlalu tinggi tingkat kesalahan. Kita tidak bisa menolak bahwa a
sama baiknya dengan b.

Uji Wilcoxon lebih tajam daripada uji ini, tetapi bergantung pada
nilai kuantitatif dari perbedaan.
\end{eulercomment}
\begin{eulerprompt}
>wilcoxon(a,b)
\end{eulerprompt}
\begin{euleroutput}
  0.0296680599405
\end{euleroutput}
\begin{eulercomment}
Mari kita mencoba dua uji lainnya menggunakan rangkaian data yang
dihasilkan.
\end{eulercomment}
\begin{eulerprompt}
>wilcoxon(normal(1,20),normal(1,20)-1)
\end{eulerprompt}
\begin{euleroutput}
  0.00499819423799
\end{euleroutput}
\begin{eulerprompt}
>wilcoxon(normal(1,20),normal(1,20))
\end{eulerprompt}
\begin{euleroutput}
  0.559353645673
\end{euleroutput}
\eulerheading{Bilangan Acak}
\begin{eulercomment}
Berikut adalah uji untuk generator bilangan acak. Euler menggunakan
generator yang sangat baik, jadi kita tidak perlu mengharapkan
masalah.

Pertama-tama kita menghasilkan sepuluh juta bilangan acak dalam
rentang [0,1].
\end{eulercomment}
\begin{eulerprompt}
>n:=10000000; r:=random(1,n);
\end{eulerprompt}
\begin{eulercomment}
Selanjutnya, kita menghitung jarak antara dua bilangan yang kurang
dari 0,05.
\end{eulercomment}
\begin{eulerprompt}
>a:=0.05; d:=differences(nonzeros(r<a));
\end{eulerprompt}
\begin{eulercomment}
Akhirnya, kita membuat plot jumlah kali setiap jarak terjadi, dan
membandingkannya dengan nilai yang diharapkan.
\end{eulercomment}
\begin{eulerprompt}
>m=getmultiplicities(1:100,d); plot2d(m); ...
>  plot2d("n*(1-a)^(x-1)*a^2",color=red,>add):
\end{eulerprompt}
\eulerimg{17}{images/EMT4Statistika_Wahyu Rananda Westri_22305144039_Matematika B-045.png}
\begin{eulercomment}
Hapus data.
\end{eulercomment}
\begin{eulerprompt}
>remvalue n;
\end{eulerprompt}
\begin{eulercomment}
\begin{eulercomment}
\eulerheading{Pengantar untuk Pengguna Proyek R}
\begin{eulercomment}
Jelas, EMT tidak bersaing dengan R sebagai paket statistik. Namun, ada
banyak prosedur statistik dan fungsi yang tersedia dalam EMT juga.
Jadi, EMT dapat memenuhi kebutuhan dasar. Pada akhirnya, EMT
dilengkapi dengan paket-paket numerik dan sistem aljabar komputer.

Buku catatan ini untuk Anda jika Anda akrab dengan R, tetapi perlu
mengetahui perbedaan dalam sintaks antara EMT dan R. Kami akan
memberikan gambaran tentang hal-hal yang jelas dan kurang jelas yang
perlu Anda ketahui.

Selain itu, kami akan mengeksplorasi cara pertukaran data antara kedua
sistem ini.
\end{eulercomment}
\begin{eulercomment}
Harap dicatat bahwa ini adalah sebuah proyek yang masih dalam proses
pengembangan.
\end{eulercomment}
\eulerheading{Sintaks Dasar}
\begin{eulercomment}
Hal pertama yang Anda pelajari dalam R adalah membuat vektor. Di EMT,
perbedaan utamanya adalah operator : dapat mengambil langkah. Selain
itu, operator : memiliki prioritas yang rendah.
\end{eulercomment}
\begin{eulerprompt}
>n=10; 0:n/20:n-1
\end{eulerprompt}
\begin{euleroutput}
  [0,  0.5,  1,  1.5,  2,  2.5,  3,  3.5,  4,  4.5,  5,  5.5,  6,  6.5,
  7,  7.5,  8,  8.5,  9]
\end{euleroutput}
\begin{eulercomment}
Fungsi c() tidak ada. Namun, Anda dapat menggunakan vektor untuk
menggabungkan hal-hal.

Contoh berikut, seperti banyak contoh lainnya, diambil dari "Pengantar
ke R" yang disertakan dalam proyek R. Jika Anda membaca PDF tersebut,
Anda akan menemukan bahwa saya mengikuti jalurnya dalam tutorial ini.
\end{eulercomment}
\begin{eulerprompt}
>x=[10.4, 5.6, 3.1, 6.4, 21.7]; [x,0,x]
\end{eulerprompt}
\begin{euleroutput}
  [10.4,  5.6,  3.1,  6.4,  21.7,  0,  10.4,  5.6,  3.1,  6.4,  21.7]
\end{euleroutput}
\begin{eulercomment}
Operator titik dua (colon operator) dengan langkah di EMT digantikan
oleh fungsi seq() di R. Kita bisa menulis fungsi ini di EMT.
\end{eulercomment}
\begin{eulerprompt}
>function seq(a,b,c) := a:b:c; ...
>seq(0,-0.1,-1)
\end{eulerprompt}
\begin{euleroutput}
  [0,  -0.1,  -0.2,  -0.3,  -0.4,  -0.5,  -0.6,  -0.7,  -0.8,  -0.9,  -1]
\end{euleroutput}
\begin{eulercomment}
Fungsi rep() dalam R tidak ada di EMT. Untuk input berupa vektor,
fungsi ini dapat ditulis sebagai berikut.
\end{eulercomment}
\begin{eulerprompt}
>function rep(x:vector,n:index) := flatten(dup(x,n)); ...
>rep(x,2)
\end{eulerprompt}
\begin{euleroutput}
  [10.4,  5.6,  3.1,  6.4,  21.7,  10.4,  5.6,  3.1,  6.4,  21.7]
\end{euleroutput}
\begin{eulercomment}
Perlu dicatat bahwa "=" atau ":=" digunakan untuk penugasan dalam EMT.
Operator "-\textgreater{}" digunakan untuk satuan dalam EMT.
\end{eulercomment}
\begin{eulerprompt}
>125km -> " miles"
\end{eulerprompt}
\begin{euleroutput}
  77.6713990297 miles
\end{euleroutput}
\begin{eulercomment}
Operator "\textless{}-" untuk penugasan memang membingungkan dan bukan ide yang
baik dalam R. Berikut akan membandingkan a dengan -4 dalam EMT.
\end{eulercomment}
\begin{eulerprompt}
>a=2; a<-4
\end{eulerprompt}
\begin{euleroutput}
  0
\end{euleroutput}
\begin{eulercomment}
Di R, "a \textless{}- 4 \textless{} 3" berfungsi, tetapi "a \textless{}- 4 \textless{}- 3" tidak. Saya juga
memiliki ambigu yang serupa di EMT, tetapi mencoba mengatasinya
sedikit demi sedikit.

EMT dan R memiliki vektor tipe boolean. Namun, di EMT, angka 0 dan 1
digunakan untuk mewakili false dan true. Di R, nilai true dan false
tetap dapat digunakan dalam aritmatika biasa seperti di EMT.
\end{eulercomment}
\begin{eulerprompt}
>x<5, %*x
\end{eulerprompt}
\begin{euleroutput}
  [0,  0,  1,  0,  0]
  [0,  0,  3.1,  0,  0]
\end{euleroutput}
\begin{eulercomment}
EMT dapat menghasilkan kesalahan atau mengembalikan NAN tergantung
pada pengaturan flag "errors".
\end{eulercomment}
\begin{eulerprompt}
>errors off; 0/0, isNAN(sqrt(-1)), errors on;
\end{eulerprompt}
\begin{euleroutput}
  NAN
  1
\end{euleroutput}
\begin{eulercomment}
String (tali) adalah sama dalam R dan EMT. Keduanya berada dalam
bahasa yang berlaku, bukan dalam Unicode.

Di R, ada paket untuk Unicode. Di EMT, sebuah string dapat menjadi
string Unicode. Sebuah string Unicode dapat diterjemahkan ke dalam
enkodean lokal dan sebaliknya. Selain itu, u"..." dapat mengandung
entitas HTML.
\end{eulercomment}
\begin{eulerprompt}
>u"&#169; Ren&eacut; Grothmann"
\end{eulerprompt}
\begin{euleroutput}
  © René Grothmann
\end{euleroutput}
\begin{eulercomment}
Berikut mungkin atau mungkin tidak ditampilkan dengan benar di sistem
Anda sebagai A dengan titik di atas dan tanda garis di atasnya. Ini
tergantung pada font yang Anda gunakan.
\end{eulercomment}
\begin{eulerprompt}
>chartoutf([480])
\end{eulerprompt}
\begin{euleroutput}
  Ǡ
\end{euleroutput}
\begin{eulercomment}
Penggabungan string dilakukan dengan "+" atau "\textbar{}". Ini dapat mencakup
angka, yang akan mencetak dalam format yang berlaku.
\end{eulercomment}
\begin{eulerprompt}
>"pi = "+pi
\end{eulerprompt}
\begin{euleroutput}
  pi = 3.14159265359
\end{euleroutput}
\eulerheading{Indeks}
\begin{eulercomment}
Sebagian besar waktu, ini akan berfungsi seperti dalam R.

Namun, EMT akan menginterpretasikan indeks negatif dari belakang
vektor, sementara R menginterpretasikan x[n] sebagai x tanpa elemen
ke-n.
\end{eulercomment}
\begin{eulerprompt}
>x, x[1:3], x[-2]
\end{eulerprompt}
\begin{euleroutput}
  [10.4,  5.6,  3.1,  6.4,  21.7]
  [10.4,  5.6,  3.1]
  6.4
\end{euleroutput}
\begin{eulercomment}
Perilaku R dapat dicapai dalam EMT dengan menggunakan drop().
\end{eulercomment}
\begin{eulerprompt}
>drop(x,2)
\end{eulerprompt}
\begin{euleroutput}
  [10.4,  3.1,  6.4,  21.7]
\end{euleroutput}
\begin{eulercomment}
Vektor logika tidak diperlakukan secara berbeda sebagai indeks dalam
EMT, berbeda dengan R. Anda perlu mengekstraksi elemen-elemen yang
bukan nol terlebih dahulu di EMT.
\end{eulercomment}
\begin{eulerprompt}
>x, x>5, x[nonzeros(x>5)]
\end{eulerprompt}
\begin{euleroutput}
  [10.4,  5.6,  3.1,  6.4,  21.7]
  [1,  1,  0,  1,  1]
  [10.4,  5.6,  6.4,  21.7]
\end{euleroutput}
\begin{eulercomment}
Sama seperti dalam R, vektor indeks dapat mengandung pengulangan.
\end{eulercomment}
\begin{eulerprompt}
>x[[1,2,2,1]]
\end{eulerprompt}
\begin{euleroutput}
  [10.4,  5.6,  5.6,  10.4]
\end{euleroutput}
\begin{eulercomment}
Tetapi penggunaan nama untuk indeks tidak mungkin di EMT. Untuk paket
statistik, ini mungkin seringkali diperlukan untuk memudahkan akses ke
elemen-elemen vektor.

Untuk meniru perilaku ini, kita dapat mendefinisikan fungsi seperti
berikut.
\end{eulercomment}
\begin{eulerprompt}
>function sel (v,i,s) := v[indexof(s,i)]; ...
>s=["first","second","third","fourth"]; sel(x,["first","third"],s)
\end{eulerprompt}
\begin{euleroutput}
  
  Trying to overwrite protected function sel!
  Error in:
  function sel (v,i,s) := v[indexof(s,i)]; ... ...
               ^
  
  Trying to overwrite protected function sel!
  Error in:
  function sel (v,i,s) := v[indexof(s,i)]; ... ...
               ^
  
  Trying to overwrite protected function sel!
  Error in:
  function sel (v,i,s) := v[indexof(s,i)]; ... ...
               ^
  [10.4,  3.1]
\end{euleroutput}
\eulerheading{Tipe Data}
\begin{eulercomment}
EMT memiliki lebih banyak tipe data yang telah ditentukan daripada R.
Dalam R, jelas terdapat vektor yang bisa tumbuh. Anda dapat mengatur
vektor numerik kosong v dan memberikan nilai ke elemen v[17]. Hal ini
tidak mungkin dilakukan di EMT.

Berikut ini sedikit tidak efisien.
\end{eulercomment}
\begin{eulerprompt}
>v=[]; for i=1 to 10000; v=v|i; end;
\end{eulerprompt}
\begin{eulercomment}
EMT akan mengkonstruksi sebuah vektor dengan v dan i yang ditambahkan
di atas tumpukan (stack) dan menyalin vektor tersebut kembali ke
variabel global v.

Cara yang lebih efisien adalah dengan mendefinisikan vektor
sebelumnya.
\end{eulercomment}
\begin{eulerprompt}
>v=zeros(10000); for i=1 to 10000; v[i]=i; end;
\end{eulerprompt}
\begin{eulercomment}
Untuk mengubah tipe data dalam EMT, Anda dapat menggunakan fungsi
seperti complex().
\end{eulercomment}
\begin{eulerprompt}
>complex(1:4)
\end{eulerprompt}
\begin{euleroutput}
  [ 1+0i ,  2+0i ,  3+0i ,  4+0i  ]
\end{euleroutput}
\begin{eulercomment}
Konversi ke string hanya mungkin untuk tipe data dasar. Format saat
ini digunakan untuk penggabungan string sederhana. Tetapi ada fungsi
seperti print() atau frac().

Untuk vektor, Anda dapat dengan mudah menulis fungsi sendiri.
\end{eulercomment}
\begin{eulerprompt}
>function tostr (v) ...
\end{eulerprompt}
\begin{eulerudf}
  s="[";
  loop 1 to length(v);
     s=s+print(v[#],2,0);
     if #<length(v) then s=s+","; endif;
  end;
  return s+"]";
  endfunction
\end{eulerudf}
\begin{eulerprompt}
>tostr(linspace(0,1,10))
\end{eulerprompt}
\begin{euleroutput}
  [0.00,0.10,0.20,0.30,0.40,0.50,0.60,0.70,0.80,0.90,1.00]
\end{euleroutput}
\begin{eulercomment}
Untuk berkomunikasi dengan Maxima, ada fungsi convertmxm(), yang juga
dapat digunakan untuk memformat vektor untuk keluaran.
\end{eulercomment}
\begin{eulerprompt}
>convertmxm(1:10)
\end{eulerprompt}
\begin{euleroutput}
  [1,2,3,4,5,6,7,8,9,10]
\end{euleroutput}
\begin{eulercomment}
Untuk LaTeX, perintah tex dapat digunakan untuk mendapatkan perintah
LaTeX.
\end{eulercomment}
\begin{eulerprompt}
>tex(&[1,2,3])
\end{eulerprompt}
\begin{euleroutput}
  \(\backslash\)left[ 1 , 2 , 3 \(\backslash\)right] 
\end{euleroutput}
\eulerheading{Faktor dan Tabel}
\begin{eulercomment}
Dalam pengantar ke R, ada contoh dengan faktor-faktor yang disebutkan.

Berikut adalah daftar wilayah dari 30 negara bagian.
\end{eulercomment}
\begin{eulerprompt}
>austates = ["tas", "sa", "qld", "nsw", "nsw", "nt", "wa", "wa", ...
>"qld", "vic", "nsw", "vic", "qld", "qld", "sa", "tas", ...
>"sa", "nt", "wa", "vic", "qld", "nsw", "nsw", "wa", ...
>"sa", "act", "nsw", "vic", "vic", "act"];
\end{eulerprompt}
\begin{eulercomment}
Misalkan, kita memiliki pendapatan yang sesuai di setiap negara
bagian.
\end{eulercomment}
\begin{eulerprompt}
>incomes = [60, 49, 40, 61, 64, 60, 59, 54, 62, 69, 70, 42, 56, ...
>61, 61, 61, 58, 51, 48, 65, 49, 49, 41, 48, 52, 46, ...
>59, 46, 58, 43];
\end{eulerprompt}
\begin{eulercomment}
Sekarang, kita ingin menghitung rata-rata pendapatan di
wilayah-wilayah tersebut. Sebagai program statistik, R memiliki fungsi
factor() dan tapply() untuk ini.

EMT dapat melakukan ini dengan menemukan indeks wilayah di daftar unik
wilayah.
\end{eulercomment}
\begin{eulerprompt}
>auterr=sort(unique(austates)); f=indexofsorted(auterr,austates)
\end{eulerprompt}
\begin{euleroutput}
  [6,  5,  4,  2,  2,  3,  8,  8,  4,  7,  2,  7,  4,  4,  5,  6,  5,  3,
  8,  7,  4,  2,  2,  8,  5,  1,  2,  7,  7,  1]
\end{euleroutput}
\begin{eulercomment}
Pada saat itu, kita dapat menulis fungsi loop sendiri untuk melakukan
sesuatu hanya untuk satu faktor saja.

Atau kita dapat meniru fungsi tapply() dengan cara berikut.
\end{eulercomment}
\begin{eulerprompt}
>function map tappl (i; f$:call, cat, x) ...
\end{eulerprompt}
\begin{eulerudf}
  u=sort(unique(cat));
  f=indexof(u,cat);
  return f$(x[nonzeros(f==indexof(u,i))]);
  endfunction
\end{eulerudf}
\begin{eulercomment}
Ini agak tidak efisien, karena menghitung wilayah-wilayah unik untuk
setiap i, tetapi ini berfungsi.
\end{eulercomment}
\begin{eulerprompt}
>tappl(auterr,"mean",austates,incomes)
\end{eulerprompt}
\begin{euleroutput}
  [44.5,  57.3333333333,  55.5,  53.6,  55,  60.5,  56,  52.25]
\end{euleroutput}
\begin{eulercomment}
Perlu dicatat bahwa ini berfungsi untuk setiap vektor wilayah.
\end{eulercomment}
\begin{eulerprompt}
>tappl(["act","nsw"],"mean",austates,incomes)
\end{eulerprompt}
\begin{euleroutput}
  [44.5,  57.3333333333]
\end{euleroutput}
\begin{eulercomment}
Sekarang, paket statistik EMT mendefinisikan tabel seperti dalam R.
Fungsi readtable() dan writetable() dapat digunakan untuk input dan
output.

Jadi kita bisa mencetak rata-rata pendapatan negara bagian di
wilayah-wilayah dengan cara yang ramah.
\end{eulercomment}
\begin{eulerprompt}
>writetable(tappl(auterr,"mean",austates,incomes),labc=auterr,wc=7)
\end{eulerprompt}
\begin{euleroutput}
      act    nsw     nt    qld     sa    tas    vic     wa
     44.5  57.33   55.5   53.6     55   60.5     56  52.25
\end{euleroutput}
\begin{eulercomment}
Kita juga dapat mencoba meniru perilaku R sepenuhnya.

Faktor-faktor harus jelas disimpan dalam koleksi dengan tipe dan
kategori (negara bagian dan wilayah dalam contoh kita). Untuk EMT,
kita tambahkan indeks yang telah dihitung sebelumnya.
\end{eulercomment}
\begin{eulerprompt}
>function makef (t) ...
\end{eulerprompt}
\begin{eulerudf}
  ## Factor data
  ## Returns a collection with data t, unique data, indices.
  ## See: tapply
  u=sort(unique(t));
  return \{\{t,u,indexofsorted(u,t)\}\};
  endfunction
\end{eulerudf}
\begin{eulerprompt}
>statef=makef(austates);
\end{eulerprompt}
\begin{eulercomment}
Sekarang, elemen ketiga dari koleksi akan berisi indeks.
\end{eulercomment}
\begin{eulerprompt}
>statef[3]
\end{eulerprompt}
\begin{euleroutput}
  [6,  5,  4,  2,  2,  3,  8,  8,  4,  7,  2,  7,  4,  4,  5,  6,  5,  3,
  8,  7,  4,  2,  2,  8,  5,  1,  2,  7,  7,  1]
\end{euleroutput}
\begin{eulercomment}
Sekarang kita bisa meniru tapply() dengan cara berikut. Ini akan
mengembalikan tabel sebagai koleksi data tabel dan judul kolom.
\end{eulercomment}
\begin{eulerprompt}
>function tapply (t:vector,tf,f$:call) ...
\end{eulerprompt}
\begin{eulerudf}
  ## Makes a table of data and factors
  ## tf : output of makef()
  ## See: makef
  uf=tf[2]; f=tf[3]; x=zeros(length(uf));
  for i=1 to length(uf);
     ind=nonzeros(f==i);
     if length(ind)==0 then x[i]=NAN;
     else x[i]=f$(t[ind]);
     endif;
  end;
  return \{\{x,uf\}\};
  endfunction
\end{eulerudf}
\begin{eulercomment}
Kami tidak menambahkan banyak pemeriksaan tipe di sini. Satu-satunya
tindakan pencegahan adalah berkaitan dengan kategori (faktor) tanpa
data. Namun, seharusnya memeriksa panjang yang benar dari t dan
kebenaran koleksi tf.

Tabel ini dapat dicetak sebagai tabel dengan writetable().
\end{eulercomment}
\begin{eulerprompt}
>writetable(tapply(incomes,statef,"mean"),wc=7)
\end{eulerprompt}
\begin{euleroutput}
      act    nsw     nt    qld     sa    tas    vic     wa
     44.5  57.33   55.5   53.6     55   60.5     56  52.25
\end{euleroutput}
\eulerheading{Arrays}
\begin{eulercomment}
EMT hanya memiliki dua dimensi untuk array. Tipe data ini disebut
matriks. Namun, akan mudah untuk menulis fungsi untuk dimensi yang
lebih tinggi atau membuat perpustakaan C untuk ini.

Di R, array adalah vektor dengan bidang dimensi.

Di EMT, sebuah vektor adalah matriks dengan satu baris. Ini dapat
diubah menjadi matriks dengan redim().
\end{eulercomment}
\begin{eulerprompt}
>shortformat; X=redim(1:20,4,5)
\end{eulerprompt}
\begin{euleroutput}
          1         2         3         4         5 
          6         7         8         9        10 
         11        12        13        14        15 
         16        17        18        19        20 
\end{euleroutput}
\begin{eulercomment}
Pengambilan baris dan kolom, atau sub-matriks, mirip dengan dalam R.
\end{eulercomment}
\begin{eulerprompt}
>X[,2:3]
\end{eulerprompt}
\begin{euleroutput}
          2         3 
          7         8 
         12        13 
         17        18 
\end{euleroutput}
\begin{eulercomment}
Namun, di R, Anda dapat mengatur daftar indeks khusus dari vektor ke
suatu nilai. Hal yang sama hanya mungkin di EMT dengan menggunakan
perulangan.
\end{eulercomment}
\begin{eulerprompt}
>function setmatrixvalue (M, i, j, v) ...
\end{eulerprompt}
\begin{eulerudf}
  loop 1 to max(length(i),length(j),length(v))
     M[i\{#\},j\{#\}] = v\{#\};
  end;
  endfunction
\end{eulerudf}
\begin{eulercomment}
Kami menunjukkan ini untuk menunjukkan bahwa matriks dilewatkan dengan
referensi di EMT. Jika Anda tidak ingin mengubah matriks asli M, Anda
perlu menyalinnya dalam fungsi.
\end{eulercomment}
\begin{eulerprompt}
>setmatrixvalue(X,1:3,3:-1:1,0); X,
\end{eulerprompt}
\begin{euleroutput}
          1         2         0         4         5 
          6         0         8         9        10 
          0        12        13        14        15 
         16        17        18        19        20 
\end{euleroutput}
\begin{eulercomment}
Produk luar (outer product) dalam EMT hanya dapat dilakukan antara
vektor. Ini dilakukan secara otomatis karena bahasa matriks. Salah
satu vektor harus menjadi vektor kolom dan yang lainnya vektor baris.
\end{eulercomment}
\begin{eulerprompt}
>(1:5)*(1:5)'
\end{eulerprompt}
\begin{euleroutput}
          1         2         3         4         5 
          2         4         6         8        10 
          3         6         9        12        15 
          4         8        12        16        20 
          5        10        15        20        25 
\end{euleroutput}
\begin{eulercomment}
Di dalam PDF pengantar untuk R terdapat contoh yang menghitung
distribusi ab-cd untuk a, b, c, d yang dipilih dari 0 hingga n secara
acak. Solusi dalam R adalah dengan menggunakan matriks 4 dimensi dan
menjalankan fungsi table() di atasnya.

Tentu saja, ini dapat dicapai dengan menggunakan perulangan. Namun,
perulangan tidak efektif di EMT atau R. Di EMT, kita bisa menulis
perulangan dalam bahasa C dan itu akan menjadi solusi yang paling
cepat.

Namun, kita ingin meniru perilaku R. Untuk ini, kita perlu meratakan
perkalian ab dan membuat matriks ab-cd.
\end{eulercomment}
\begin{eulerprompt}
>a=0:6; b=a'; p=flatten(a*b); q=flatten(p-p'); ...
>u=sort(unique(q)); f=getmultiplicities(u,q); ...
>statplot(u,f,"h"):
\end{eulerprompt}
\eulerimg{17}{images/EMT4Statistika_Wahyu Rananda Westri_22305144039_Matematika B-046.png}
\begin{eulercomment}
Selain jumlah yang tepat, EMT dapat menghitung frekuensi dalam vektor.
\end{eulercomment}
\begin{eulerprompt}
>getfrequencies(q,-50:10:50)
\end{eulerprompt}
\begin{euleroutput}
  [0,  23,  132,  316,  602,  801,  333,  141,  53,  0]
\end{euleroutput}
\begin{eulercomment}
Cara paling mudah untuk menggambarkannya sebagai distribusi adalah
sebagai berikut.
\end{eulercomment}
\begin{eulerprompt}
>plot2d(q,distribution=11):
\end{eulerprompt}
\eulerimg{17}{images/EMT4Statistika_Wahyu Rananda Westri_22305144039_Matematika B-047.png}
\begin{eulercomment}
Tetapi juga mungkin untuk menghitung jumlah dalam interval yang telah
dipilih sebelumnya. Tentu saja, berikut ini menggunakan
getfrequencies() secara internal.

Karena fungsi histo() mengembalikan frekuensi, kita perlu menyesuaikan
sehingga integral di bawah grafik batangnya adalah 1.
\end{eulercomment}
\begin{eulerprompt}
>\{x,y\}=histo(q,v=-55:10:55); y=y/sum(y)/differences(x); ...
>plot2d(x,y,>bar,style="/"):
\end{eulerprompt}
\eulerimg{17}{images/EMT4Statistika_Wahyu Rananda Westri_22305144039_Matematika B-048.png}
\eulerheading{Lists}
\begin{eulercomment}
EMT memiliki dua jenis daftar. Satu adalah daftar global yang dapat
diubah, dan yang lainnya adalah tipe daftar yang tidak dapat diubah.
Kami tidak membahas tentang daftar global di sini.

Tipe daftar yang tidak dapat diubah disebut koleksi dalam EMT. Ini
berperilaku seperti struktur dalam C, tetapi elemennya hanya diberi
nomor dan tidak dinamai.
\end{eulercomment}
\begin{eulerprompt}
>L=\{\{"Fred","Flintstone",40,[1990,1992]\}\}
\end{eulerprompt}
\begin{euleroutput}
  Fred
  Flintstone
  40
  [1990,  1992]
\end{euleroutput}
\begin{eulercomment}
Saat ini, elemen-elemen tidak memiliki nama, meskipun nama dapat
diatur untuk tujuan khusus. Mereka diakses dengan nomor.
\end{eulercomment}
\begin{eulerprompt}
>(L[4])[2]
\end{eulerprompt}
\begin{euleroutput}
  1992
\end{euleroutput}
\begin{eulercomment}
\begin{eulercomment}
\eulerheading{Input dan Output Berkas (Membaca dan Menulis Data)}
\begin{eulercomment}
Anda seringkali akan ingin mengimpor matriks data dari sumber lain ke
EMT. Panduan ini memberi tahu Anda tentang banyak cara untuk mencapai
ini. Fungsi sederhana adalah writematrix() dan readmatrix().

Mari kita tunjukkan bagaimana cara membaca dan menulis vektor bilangan
riil ke berkas.
\end{eulercomment}
\begin{eulerprompt}
>a=random(1,100); mean(a), dev(a),
\end{eulerprompt}
\begin{euleroutput}
  0.54004
  0.28329
\end{euleroutput}
\begin{eulercomment}
Untuk menulis data ke berkas, kita menggunakan fungsi writematrix().

Karena pengantar ini kemungkinan besar berada di direktori di mana
pengguna tidak memiliki izin menulis, kita menulis data ke direktori
beranda pengguna. Untuk notebook Anda sendiri, hal ini tidak perlu,
karena berkas data akan ditulis ke dalam direktori yang sama.
\end{eulercomment}
\begin{eulerprompt}
>filename="test.dat";
\end{eulerprompt}
\begin{eulercomment}
Sekarang kita menulis vektor kolom \textbackslash{}( \textbackslash{}mathbf\{a'\} \textbackslash{}) ke dalam file.
Hal ini akan menghasilkan satu angka di setiap baris file tersebut.
\end{eulercomment}
\begin{eulerprompt}
>writematrix(a',filename);
\end{eulerprompt}
\begin{eulercomment}
Untuk membaca data tersebut, kita menggunakan fungsi readmatrix().
\end{eulercomment}
\begin{eulerprompt}
>a=readmatrix(filename)';
\end{eulerprompt}
\begin{eulercomment}
Dan menghapus file tersebut.
\end{eulercomment}
\begin{eulerprompt}
>fileremove(filename);
>mean(a), dev(a),
\end{eulerprompt}
\begin{euleroutput}
  0.54004
  0.28329
\end{euleroutput}
\begin{eulercomment}
Fungsi writematrix() atau writetable() dapat dikonfigurasi untuk
bahasa lain.

Contohnya, jika Anda menggunakan sistem berbahasa Indonesia (titik
desimal digantikan oleh koma), Excel Anda memerlukan nilai dengan koma
desimal yang dipisahkan oleh titik koma dalam file csv (yang secara
default dipisahkan oleh koma). File berikut "test.csv" akan muncul di
folder Anda saat ini.
\end{eulercomment}
\begin{eulerprompt}
>filename="test.csv"; ...
>writematrix(random(5,3),file=filename,separator=",");
\end{eulerprompt}
\begin{eulercomment}
Anda sekarang dapat membuka file ini langsung dengan Excel berbahasa
Indonesia.
\end{eulercomment}
\begin{eulerprompt}
>fileremove(filename);
\end{eulerprompt}
\begin{eulercomment}
Terkadang kita memiliki string dengan token-token seperti contoh
berikut.
\end{eulercomment}
\begin{eulerprompt}
>s1:="f m m f m m m f f f m m f";  ...
>s2:="f f f m m f f";
\end{eulerprompt}
\begin{eulercomment}
Untuk mengonversi ini menjadi token, kita akan mendefinisikan sebuah
vektor dari token-token tersebut.
\end{eulercomment}
\begin{eulerprompt}
>tok:=["f","m"]
\end{eulerprompt}
\begin{euleroutput}
  f
  m
\end{euleroutput}
\begin{eulercomment}
Kemudian kita dapat menghitung berapa kali setiap token muncul dalam
string tersebut, dan memasukkan hasilnya ke dalam sebuah tabel.
\end{eulercomment}
\begin{eulerprompt}
>M:=getmultiplicities(tok,strtokens(s1))_ ...
>  getmultiplicities(tok,strtokens(s2));
\end{eulerprompt}
\begin{eulercomment}
Menuliskan tabel dengan judul token.
\end{eulercomment}
\begin{eulerprompt}
>writetable(M,labc=tok,labr=1:2,wc=8)
\end{eulerprompt}
\begin{euleroutput}
                 f       m
         1       6       7
         2       5       2
\end{euleroutput}
\begin{eulercomment}
Untuk statistik, EMT dapat membaca dan menulis tabel.
\end{eulercomment}
\begin{eulerprompt}
>file="test.dat"; open(file,"w"); ...
>writeln("A,B,C"); writematrix(random(3,3)); ...
>close();
\end{eulerprompt}
\begin{eulercomment}
File tersebut terlihat seperti ini.
\end{eulercomment}
\begin{eulerprompt}
>printfile(file)
\end{eulerprompt}
\begin{euleroutput}
  A,B,C
  0.3084163350124789,0.7860228753704319,0.3640480404459329
  0.3235527726365187,0.5355164285371529,0.4209486538454125
  0.9647906143251817,0.4305557169007305,0.8295841224094156
  
\end{euleroutput}
\begin{eulercomment}
Fungsi readtable() dalam bentuk paling sederhana dapat membaca ini dan
mengembalikan kumpulan nilai dan baris judul.
\end{eulercomment}
\begin{eulerprompt}
>L=readtable(file,>list);
\end{eulerprompt}
\begin{eulercomment}
Kumpulan ini dapat dicetak menggunakan writetable() ke buku catatan,
atau ke dalam sebuah file.
\end{eulercomment}
\begin{eulerprompt}
>writetable(L,wc=10,dc=5)
\end{eulerprompt}
\begin{euleroutput}
           A         B         C
     0.30842   0.78602   0.36405
     0.32355   0.53552   0.42095
     0.96479   0.43056   0.82958
\end{euleroutput}
\begin{eulercomment}
Matriks nilai adalah elemen pertama dari L. Perlu diperhatikan bahwa
fungsi mean() dalam EMT menghitung nilai rata-rata dari baris-baris
dalam suatu matriks.
\end{eulercomment}
\begin{eulerprompt}
>mean(L[1])
\end{eulerprompt}
\begin{euleroutput}
    0.48616 
    0.42667 
    0.74164 
\end{euleroutput}
\begin{eulercomment}
*File CSV*

Pertama, mari tulis sebuah matriks ke dalam sebuah file. Untuk output,
kita akan membuat file dalam direktori kerja saat ini.
\end{eulercomment}
\begin{eulerprompt}
>file="test.csv";  ...
>M=random(3,3); writematrix(M,file);
\end{eulerprompt}
\begin{eulercomment}
Berikut adalah konten dari file tersebut.
\end{eulercomment}
\begin{eulerprompt}
>printfile(file)
\end{eulerprompt}
\begin{euleroutput}
  0.4985755441612197,0.8908902130674888,0.230993153822803
  0.5388022720805338,0.03150264484701902,0.9359045715778547
  0.6011875635483036,0.1012503400474223,0.4840335655691349
  
\end{euleroutput}
\begin{eulercomment}
File CSV ini dapat dibuka pada sistem berbahasa Inggris di Excel
dengan mengklik dua kali. Jika Anda mendapatkan file seperti ini di
sistem berbahasa Jerman, Anda perlu mengimpor data ke dalam Excel
dengan memperhatikan tanda titik desimal.

Namun, tanda titik desimal adalah format default untuk EMT juga. Anda
dapat membaca sebuah matriks dari sebuah file dengan menggunakan
fungsi readmatrix().
\end{eulercomment}
\begin{eulerprompt}
>readmatrix(file)
\end{eulerprompt}
\begin{euleroutput}
    0.49858   0.89089   0.23099 
     0.5388  0.031503    0.9359 
    0.60119   0.10125   0.48403 
\end{euleroutput}
\begin{eulercomment}
Memungkinkan untuk menulis beberapa matriks ke dalam satu file.
Perintah open() dapat membuka sebuah file untuk penulisan dengan
parameter "w". Nilai default adalah "r" untuk membaca.
\end{eulercomment}
\begin{eulerprompt}
>open(file,"w"); writematrix(M); writematrix(M'); close();
\end{eulerprompt}
\begin{eulercomment}
Matriks-matriks tersebut dipisahkan oleh sebuah baris kosong. Untuk
membaca matriks-matriks tersebut, buka file dan panggil fungsi
readmatrix() beberapa kali.
\end{eulercomment}
\begin{eulerprompt}
>open(file); A=readmatrix(); B=readmatrix(); A==B, close();
\end{eulerprompt}
\begin{euleroutput}
          1         0         0 
          0         1         0 
          0         0         1 
\end{euleroutput}
\begin{eulercomment}
Di Excel atau spreadsheet serupa, Anda dapat mengekspor sebuah matriks
sebagai CSV (comma separated values). Di Excel 2007, gunakan "save as"
dan "other formats", lalu pilih "CSV". Pastikan bahwa tabel saat ini
hanya berisi data yang ingin Anda ekspor.

Berikut adalah contohnya.
\end{eulercomment}
\begin{eulerprompt}
>printfile("excel-data.csv")
\end{eulerprompt}
\begin{euleroutput}
  Could not open the file
  excel-data.csv
  for reading!
  Try "trace errors" to inspect local variables after errors.
  printfile:
      open(filename,"r");
\end{euleroutput}
\begin{eulercomment}
Seperti yang dapat Anda lihat, sistem Jerman saya menggunakan titik
koma sebagai pemisah dan tanda desimal koma. Anda dapat mengubah
pengaturan ini dalam pengaturan sistem atau di Excel, tetapi ini tidak
diperlukan untuk membaca matriks ke dalam EMT.

Cara paling mudah untuk membaca ini ke dalam Euler adalah menggunakan
readmatrix(). Semua koma akan digantikan oleh titik dengan menggunakan
parameter \textgreater{}comma. Untuk CSV berbahasa Inggris, cukup abaikan parameter
ini.
\end{eulercomment}
\begin{eulerprompt}
>M=readmatrix("excel-data.csv",>comma)
\end{eulerprompt}
\begin{euleroutput}
  Could not open the file
  excel-data.csv
  for reading!
  Try "trace errors" to inspect local variables after errors.
  readmatrix:
      if filename<>"" then open(filename,"r"); endif;
\end{euleroutput}
\begin{eulercomment}
Mari memplotnya.
\end{eulercomment}
\begin{eulerprompt}
>plot2d(M'[1],M'[2:3],>points,color=[red,green]'):
\end{eulerprompt}
\eulerimg{17}{images/EMT4Statistika_Wahyu Rananda Westri_22305144039_Matematika B-049.png}
\begin{eulercomment}
Ada cara-cara dasar lain untuk membaca data dari sebuah file. Anda
dapat membuka file dan membaca angka-angka satu per satu dari setiap
baris. Fungsi getvectorline() akan membaca angka-angka dari sebuah
baris data. Secara default, fungsi ini mengharapkan titik desimal.
Namun, Anda juga dapat menggunakan koma desimal dengan memanggil
setdecimaldot(",") sebelum Anda menggunakan fungsi ini.

Berikut adalah contoh fungsi untuk membaca data tersebut. Fungsi ini
akan berhenti pada akhir file atau baris kosong.
\end{eulercomment}
\begin{eulerprompt}
>function myload (file) ...
\end{eulerprompt}
\begin{eulerudf}
  open(file);
  M=[];
  repeat
     until eof();
     v=getvectorline(3);
     if length(v)>0 then M=M_v; else break; endif;
  end;
  return M;
  close(file);
  endfunction
\end{eulerudf}
\begin{eulerprompt}
>myload(file)
\end{eulerprompt}
\begin{euleroutput}
    0.49858         0   0.89089         0   0.23099 
     0.5388         0  0.031503         0    0.9359 
    0.60119         0   0.10125         0   0.48403 
\end{euleroutput}
\begin{eulercomment}
Juga memungkinkan untuk membaca semua angka dalam file tersebut
menggunakan fungsi getvector().
\end{eulercomment}
\begin{eulerprompt}
>open(file); v=getvector(10000); close(); redim(v[1:9],3,3)
\end{eulerprompt}
\begin{euleroutput}
    0.49858         0   0.89089 
          0   0.23099    0.5388 
          0  0.031503         0 
\end{euleroutput}
\begin{eulercomment}
Dengan demikian, sangat mudah untuk menyimpan vektor nilai, satu nilai
dalam setiap baris, dan membaca kembali vektor ini.
\end{eulercomment}
\begin{eulerprompt}
>v=random(1000); mean(v)
\end{eulerprompt}
\begin{euleroutput}
  0.49726
\end{euleroutput}
\begin{eulerprompt}
>writematrix(v',file); mean(readmatrix(file)')
\end{eulerprompt}
\begin{euleroutput}
  0.49726
\end{euleroutput}
\eulerheading{Penggunaan Tabel}
\begin{eulercomment}
Tabel dapat digunakan untuk membaca atau menulis data numerik. Sebagai
contoh, kita akan menulis sebuah tabel dengan judul baris dan kolom ke
dalam sebuah file.
\end{eulercomment}
\begin{eulerprompt}
>file="test.tab"; M=random(3,3);  ...
>open(file,"w");  ...
>writetable(M,separator=",",labc=["one","two","three"]);  ...
>close(); ...
>printfile(file)
\end{eulerprompt}
\begin{euleroutput}
  one,two,three
        0.58,      0.69,      0.89
        0.88,      0.43,      0.71
        0.18,         1,      0.35
\end{euleroutput}
\begin{eulercomment}
Ini dapat diimpor ke dalam Excel.

Untuk membaca file tersebut di EMT, kita menggunakan fungsi
readtable().
\end{eulercomment}
\begin{eulerprompt}
>\{M,headings\}=readtable(file,>clabs); ...
>writetable(M,labc=headings)
\end{eulerprompt}
\begin{euleroutput}
         one       two     three
        0.58      0.69      0.89
        0.88      0.43      0.71
        0.18         1      0.35
\end{euleroutput}
\eulerheading{Menganalisis Sebuah Baris}
\begin{eulercomment}
Anda bahkan dapat mengevaluasi setiap baris secara manual. Misalkan,
kita memiliki sebuah baris dengan format berikut.
\end{eulercomment}
\begin{eulerprompt}
>line="2020-11-03,Tue,1'114.05"
\end{eulerprompt}
\begin{euleroutput}
  2020-11-03,Tue,1'114.05
\end{euleroutput}
\begin{eulercomment}
Pertama, kita dapat melakukan tokenisasi terhadap baris tersebut.
\end{eulercomment}
\begin{eulerprompt}
>vt=strtokens(line)
\end{eulerprompt}
\begin{euleroutput}
  2020-11-03
  Tue
  1'114.05
\end{euleroutput}
\begin{eulercomment}
Kemudian, kita dapat mengevaluasi setiap elemen dari baris tersebut
menggunakan evaluasi yang sesuai.
\end{eulercomment}
\begin{eulerprompt}
>day(vt[1]),  ...
>indexof(["mon","tue","wed","thu","fri","sat","sun"],tolower(vt[2])),  ...
>strrepl(vt[3],"'","")()
\end{eulerprompt}
\begin{euleroutput}
  7.3816e+05
  2
  1114
\end{euleroutput}
\begin{eulercomment}
Dengan menggunakan regular expressions, mungkin untuk mengekstrak
hampir semua informasi dari sebuah baris data.

Misalkan kita memiliki baris berikut dalam sebuah dokumen HTML.
\end{eulercomment}
\begin{eulerprompt}
>line="<tr><td>1145.45</td><td>5.6</td><td>-4.5</td><tr>"
\end{eulerprompt}
\begin{euleroutput}
  <tr><td>1145.45</td><td>5.6</td><td>-4.5</td><tr>
\end{euleroutput}
\begin{eulercomment}
Untuk mengekstrak ini, kita menggunakan ekspresi reguler, yang
mencari:

- tanda kurung penutup \textgreater{},\\
- string apa pun yang tidak mengandung tanda kurung dengan
sub-pencocokan "(...)",\\
- tanda kurung buka dan tanda kurung tutup dengan solusi terpendek,\\
- lagi string apa pun yang tidak mengandung tanda kurung,\\
- dan tanda kurung buka \textless{}.

Ekspresi reguler agak sulit untuk dipelajari tetapi sangat kuat.
\end{eulercomment}
\begin{eulerprompt}
>\{pos,s,vt\}=strxfind(line,">([^<>]+)<.+?>([^<>]+)<");
\end{eulerprompt}
\begin{eulercomment}
The result is the position of the match, the matched string, and a
vector of strings for sub-matches.
\end{eulercomment}
\begin{eulerprompt}
>for k=1:length(vt); vt[k](), end;
\end{eulerprompt}
\begin{euleroutput}
  1145.5
  5.6
\end{euleroutput}
\begin{eulercomment}
Berikut adalah contoh fungsi yang membaca semua item numerik antara
\textless{}td\textgreater{} dan \textless{}/td\textgreater{}.
\end{eulercomment}
\begin{eulerprompt}
>function readtd (line) ...
\end{eulerprompt}
\begin{eulerudf}
  v=[]; cp=0;
  repeat
     \{pos,s,vt\}=strxfind(line,"<td.*?>(.+?)</td>",cp);
     until pos==0;
     if length(vt)>0 then v=v|vt[1]; endif;
     cp=pos+strlen(s);
  end;
  return v;
  endfunction
\end{eulerudf}
\begin{eulerprompt}
>readtd(line+"<td>non-numerical</td>")
\end{eulerprompt}
\begin{euleroutput}
  1145.45
  5.6
  -4.5
  non-numerical
\end{euleroutput}
\eulerheading{Membaca dari Web}
\begin{eulercomment}
Sebuah situs web atau file dengan URL dapat dibuka di EMT dan dapat
dibaca baris per baris.

Pada contoh ini, kita membaca versi terbaru dari situs EMT. Kita
menggunakan ekspresi reguler untuk mencari "Versi ..." dalam judul.
\end{eulercomment}
\begin{eulerprompt}
>function readversion () ...
\end{eulerprompt}
\begin{eulerudf}
  urlopen("http://www.euler-math-toolbox.de/Programs/Changes.html");
  repeat
    until urleof();
    s=urlgetline();
    k=strfind(s,"Version ",1);
    if k>0 then substring(s,k,strfind(s,"<",k)-1), break; endif;
  end;
  urlclose();
  endfunction
\end{eulerudf}
\begin{eulerprompt}
>readversion
\end{eulerprompt}
\begin{euleroutput}
  Version 2022-05-18
\end{euleroutput}
\eulerheading{Input dan Output Variabel}
\begin{eulercomment}
Anda dapat menulis sebuah variabel dalam bentuk definisi Euler ke
sebuah file atau ke baris perintah.
\end{eulercomment}
\begin{eulerprompt}
>writevar(pi,"mypi");
\end{eulerprompt}
\begin{euleroutput}
  mypi = 3.141592653589793;
\end{euleroutput}
\begin{eulercomment}
Untuk uji coba, kita akan menghasilkan sebuah file Euler di direktori
kerja EMT.
\end{eulercomment}
\begin{eulerprompt}
>file="test.e"; ...
>writevar(random(2,2),"M",file); ...
>printfile(file,3)
\end{eulerprompt}
\begin{euleroutput}
  M = [ ..
  0.1715864118118049, 0.7902075714281032;
  0.3235267463604796, 0.08344844686478282];
\end{euleroutput}
\begin{eulercomment}
Sekarang kita dapat memuat file tersebut. Ini akan mendefinisikan
matriks M.
\end{eulercomment}
\begin{eulerprompt}
>load(file); show M,
\end{eulerprompt}
\begin{euleroutput}
  M = 
    0.17159   0.79021 
    0.32353  0.083448 
\end{euleroutput}
\begin{eulercomment}
Sebagai informasi tambahan, jika fungsi writevar() digunakan pada
sebuah variabel, itu akan mencetak definisi variabel dengan nama
variabel tersebut.
\end{eulercomment}
\begin{eulerprompt}
>writevar(M); writevar(inch$)
\end{eulerprompt}
\begin{euleroutput}
  M = [ ..
  0.1715864118118049, 0.7902075714281032;
  0.3235267463604796, 0.08344844686478282];
  inch$ = 0.0254;
\end{euleroutput}
\begin{eulercomment}
Kita juga dapat membuka sebuah file baru atau menambahkan ke dalam
file yang sudah ada. Pada contoh ini, kita menambahkan ke dalam file
yang telah dihasilkan sebelumnya.
\end{eulercomment}
\begin{eulerprompt}
>open(file,"a"); ...
>writevar(random(2,2),"M1"); ...
>writevar(random(3,1),"M2"); ...
>close();
>load(file); show M1; show M2;
\end{eulerprompt}
\begin{euleroutput}
  M1 = 
    0.64914   0.13714 
      0.112   0.43951 
  M2 = 
    0.77904 
    0.52075 
    0.88929 
\end{euleroutput}
\begin{eulercomment}
Untuk menghapus file-file, gunakan fungsi fileremove().
\end{eulercomment}
\begin{eulerprompt}
>fileremove(file);
\end{eulerprompt}
\begin{eulercomment}
Sebuah vektor baris dalam sebuah file tidak memerlukan koma jika
setiap angka berada di baris baru. Mari kita hasilkan file seperti
itu, menulis setiap baris satu per satu dengan menggunakan writeln().
\end{eulercomment}
\begin{eulerprompt}
>open(file,"w"); writeln("M = ["); ...
>for i=1 to 5; writeln(""+random()); end; ...
>writeln("];"); close(); ...
>printfile(file)
\end{eulerprompt}
\begin{euleroutput}
  M = [
  0.441463853011
  0.602559586157
  0.8008250194
  0.624852131639
  0.53481766277
  ];
\end{euleroutput}
\begin{eulerprompt}
>load(file); M
\end{eulerprompt}
\begin{euleroutput}
  [0.44146,  0.60256,  0.80083,  0.62485,  0.53482]
\end{euleroutput}
\eulerheading{Contoh Soal}
\eulersubheading{Contoh Soal 1}
\begin{eulercomment}
Banyaknya perawat di 6 klinik adalah 3,5,6,4,5, dan 6. Dengan
memandang data itu sebagai data populasi, hitunglah nilai rata-rata
banyaknya perawat di 6 klinik tersebut!\\
Penyelesaian:
\end{eulercomment}
\begin{eulerprompt}
>x=[3,5,6,4,5,6]; mean(x),
\end{eulerprompt}
\begin{euleroutput}
  4.8333
\end{euleroutput}
\begin{eulercomment}
Jadi, nilai rata-rata banyaknya pegawai di 6 klinik tersebut adalah
4.83

\end{eulercomment}
\eulersubheading{Contoh Soal 2}
\begin{eulercomment}
1. Data berikut menunjukkan tinggi badan dari 20 siswa SMA 2 Surabaya.\\
Siswa yang tinggi badannya dalam rentang 145-150 sebanyak 1 orang,
dalam rentang 151-155 sebanyak 2 orang, dalam rentang 156-160 sebanyak
4 orang, dalam rentang 161-165 sebanyak 3 orang, dalam rentang 166-170
sebanyak 3 orang, dalam rentang 171-175 sebanyak 4 orang, dan dalam
rentang 176-180 sebanyak 3 orang.\\
Tentukan rata-rata tinggi badan dari 20 siswa tersebut!\\
Penyelesaian:\\
Menentukan tepi bawah kelas yang terkecil
\end{eulercomment}
\begin{eulerprompt}
>146-0.5
\end{eulerprompt}
\begin{euleroutput}
  145.5
\end{euleroutput}
\begin{eulercomment}
Menentukan panjang kelas
\end{eulercomment}
\begin{eulerprompt}
>(150-146)+1
\end{eulerprompt}
\begin{euleroutput}
  5
\end{euleroutput}
\begin{eulercomment}
Menentukan tepi atas kelas yang terbesar
\end{eulercomment}
\begin{eulerprompt}
>180+0.5
\end{eulerprompt}
\begin{euleroutput}
  180.5
\end{euleroutput}
\begin{eulerprompt}
>r=145.5:5:180.5; v=[1,2,4,3,3,4,3];
>T:=r[1:7]' | r[2:8]' | v'; writetable(T,labc=["TB","TA","Frek"])
\end{eulerprompt}
\begin{euleroutput}
          TB        TA      Frek
       145.5     150.5         1
       150.5     155.5         2
       155.5     160.5         4
       160.5     165.5         3
       165.5     170.5         3
       170.5     175.5         4
       175.5     180.5         3
\end{euleroutput}
\begin{eulercomment}
Menentukan titik tengah
\end{eulercomment}
\begin{eulerprompt}
>(T[,1]+T[,2])/2 //titik tengah dari tiap interval
\end{eulerprompt}
\begin{euleroutput}
        148 
        153 
        158 
        163 
        168 
        173 
        178 
\end{euleroutput}
\begin{eulerprompt}
>t=fold(r,[0.5,0.5])
\end{eulerprompt}
\begin{euleroutput}
  [148,  153,  158,  163,  168,  173,  178]
\end{euleroutput}
\begin{eulerprompt}
>mean(t,v)
\end{eulerprompt}
\begin{euleroutput}
  165.25
\end{euleroutput}
\begin{eulercomment}
Jadi, nilai rata-rata tinggi badan dari 20 siswa tersebut adalah
165.25.

\end{eulercomment}
\eulersubheading{Contoh 3}
\begin{eulercomment}
Misalnya kita akan menghitung nilai rata-rata(mean) yang terdapat
dalam file "test.dat"
\end{eulercomment}
\begin{eulerprompt}
>filename="test.dat"; ...
>V=random(3,3); writematrix(V,filename);
>printfile(filename),
\end{eulerprompt}
\begin{euleroutput}
  0.1858294139595237,0.2359542823657924,0.2704760003790539
  0.82636390358253,0.156173368112527,0.9576388301324568
  0.04846986155954107,0.8722870876753714,0.5779228155144673
  
\end{euleroutput}
\begin{eulerprompt}
>readmatrix(filename)
\end{eulerprompt}
\begin{euleroutput}
    0.18583         0   0.23595         0   0.27048 
    0.82636         0   0.15617         0   0.95764 
    0.04847         0   0.87229         0   0.57792 
\end{euleroutput}
\begin{eulerprompt}
>mean(V),
\end{eulerprompt}
\begin{euleroutput}
    0.23075 
    0.64673 
    0.49956 
\end{euleroutput}
\eulersubheading{Contoh 4}
\begin{eulercomment}
Disajikan data urut yaitu
45,48,49,50,52,52,52,53,53,54,54,54,54,54,56,56,
56,56,57,57,58,58,58,58,58,58,58,59,59,60,60,60,
62,62,62,63,63,64,64,65,67,68,69,70,70,71,73,74.\\
Buatlah distribusi frekuensi berdasarkan data diatas!\\
Penyeleaian:\\
\end{eulercomment}
\begin{eulerttcomment}
         - Menentukan range
           range= nilai maks-nilai min
                = 74-45
                = 29
         - Menentukan banyak kelas dengan aturan
           struges.
           = 1+3,3 log n, n banyaknya data
           = 1+3,3 log 48
           = 6,64
           = 7
         - Menentukan panjang kelas
\end{eulerttcomment}
\begin{eulerformula}
\[
p=\frac {range}{banyak kelas}
\]
\end{eulerformula}
\begin{eulerformula}
\[
p=\frac {29}{7}
\]
\end{eulerformula}
\begin{eulerformula}
\[
p= 4.14=5
\]
\end{eulerformula}
\begin{eulercomment}
Berdasarkan pertimbangan beberapa unsur dalam data urut diatas yaitu
nilai minimum 45, nilai maksimum 74, banyak kelas yaitu 7, dan panjang
kelas yaitu 5 maka dapat dibuat tabel distribusi frekuensi dengan
batas bawah kelas pertama yaitu 43 dan batas atas kelas ketujuh yaitu
77. Sehingga dapat ditentukan tepi bawah kelas pertama yaitu
43-0.5=42.5 dan tepi atas kelas ketujuh yaitu 77+0.5=77.5.
\end{eulercomment}
\begin{eulerprompt}
>r=42.5:5:77.5; v=[1,6,13,15,6,5,2];
>T:=r[1:7]' | r[2:8]' | v'; writetable(T,labc=["TB","TA","Frek"])
\end{eulerprompt}
\begin{euleroutput}
          TB        TA      Frek
        42.5      47.5         1
        47.5      52.5         6
        52.5      57.5        13
        57.5      62.5        15
        62.5      67.5         6
        67.5      72.5         5
        72.5      77.5         2
\end{euleroutput}
\begin{eulercomment}
Mencari titik tengah
\end{eulercomment}
\begin{eulerprompt}
>(T[,1]+T[,2])/2 // titik tengah tiap interval
\end{eulerprompt}
\begin{euleroutput}
         45 
         50 
         55 
         60 
         65 
         70 
         75 
\end{euleroutput}
\begin{eulercomment}
Sajian dalam bentuk histogram
\end{eulercomment}
\begin{eulerprompt}
>plot2d(r,v,a=40,b=80,c=0,d=20,bar=1,style="\(\backslash\)/"):
\end{eulerprompt}
\eulerimg{17}{images/EMT4Statistika_Wahyu Rananda Westri_22305144039_Matematika B-051.png}
\eulersubheading{Contoh 5}
\begin{eulercomment}
Berikut daftar ukuran sepatu anak kelas 5 SDN 1 Makassar.\\
39,35,35,36,37,38,42,40,38,41,37,35,38,40,41,40.\\
Tentukan median dari data tersebut!\\
Penyelesaian :
\end{eulercomment}
\begin{eulerprompt}
>data=[39,35,35,36,37,38,42,40,38,41,37,35,38,40,41,40];
>urut=sort(data)
\end{eulerprompt}
\begin{euleroutput}
  [35,  35,  35,  36,  37,  37,  38,  38,  38,  39,  40,  40,  40,  41,
  41,  42]
\end{euleroutput}
\begin{eulerprompt}
>median(data)
\end{eulerprompt}
\begin{euleroutput}
  38
\end{euleroutput}
\begin{eulercomment}
Jadi, median dari data tersebut adalah 38.

\end{eulercomment}
\eulersubheading{Contoh 6}
\begin{eulercomment}
Berikut adalah data hasil dari pengukuran berat badan 50 siswa SDN 4
Banten. Siswa yang mempunyai berat badan dalam rentang 21-26 kg
sebanyak 6 orang, yang mempunyai berat badan dalam rentang 27-32 kg
sebanyak 9 orang, yang mempunyai berat badan dalam rentang 33-38 kg
sebanyak 14 orang, yang mempunyai berat badan dalam rentang 39-44 kg
sebanyak 12 orang, yang mempunyai berat badan dalam rentang 45-50 kg
sebanyak 7 orang, dan yang mempunyai berat badan 51-56 kg sebanyak 2
orang. Tentukan median dari data hasil pengukuran berat badan 50 siswa
di SD tersebut!\\
Penyelesaian:\\
Menentukan tepi bawah kelas yang terkecil
\end{eulercomment}
\begin{eulerprompt}
>21-0.5
\end{eulerprompt}
\begin{euleroutput}
  20.5
\end{euleroutput}
\begin{eulercomment}
Menentukan panjang kelas
\end{eulercomment}
\begin{eulerprompt}
>(26-21)+1
\end{eulerprompt}
\begin{euleroutput}
  6
\end{euleroutput}
\begin{eulercomment}
Menentukan tepi atas kelas yang terbesar
\end{eulercomment}
\begin{eulerprompt}
>56+0.5
\end{eulerprompt}
\begin{euleroutput}
  56.5
\end{euleroutput}
\begin{eulerprompt}
>r=20.5:6:56.5; v=[6,9,14,12,7,2];
>T:=r[1:6]' | r[2:7]' | v'; writetable(T,labc=["TB","TA","frek"])
\end{eulerprompt}
\begin{euleroutput}
          TB        TA      frek
        20.5      26.5         6
        26.5      32.5         9
        32.5      38.5        14
        38.5      44.5        12
        44.5      50.5         7
        50.5      56.5         2
\end{euleroutput}
\begin{eulercomment}
Berdasarkan data, median berada pada urutan ke 25, maka median berada
pada kelas 32.5-38.5.
\end{eulercomment}
\begin{eulerprompt}
>Tb=32.5, p=6, n=50, Fks=15, fm=14
\end{eulerprompt}
\begin{euleroutput}
  32.5
  6
  50
  15
  14
\end{euleroutput}
\begin{eulerprompt}
>Tb+p*(1/2*n-Fks)/fm
\end{eulerprompt}
\begin{euleroutput}
  36.786
\end{euleroutput}
\begin{eulercomment}
Jadi, median dari data hasil pengukuran berat badan 50 siswa SDN 4
Banten adalah 36.785.

\end{eulercomment}
\eulersubheading{Contoh 7}
\begin{eulercomment}
Berikut adalah data hasil dari pengukuran berat badan 30 siswa SDN 5
Jember. Siswa yang mempunyai berat badan dalam rentang 21-25 kg
sebanyak 1 orang, yang mempunyai berat badan dalam rentang 26-30 kg
sebanyak 8 orang, yang mempunyai berat badan dalam rentang 31-35 kg
sebanyak 10 orang, yang mempunyai berat badan dalam rentang 36-40 kg
sebanyak 5 orang, yang mempunyai berat badan dalam rentang 41-45 kg
sebanyak 4 orang, dan yang mempunyai berat badan 46-50 kg sebanyak 2
orang. Tentukan modus dari data hasil pengukuran berat badan 30 siswa
di SD tersebut!\\
Penyelesaian:\\
Menentukan tepi bawah kelas yang terkecil
\end{eulercomment}
\begin{eulerprompt}
>21-0.5
\end{eulerprompt}
\begin{euleroutput}
  20.5
\end{euleroutput}
\begin{eulercomment}
Menentukan panjang kelas
\end{eulercomment}
\begin{eulerprompt}
>(25-21)+1
\end{eulerprompt}
\begin{euleroutput}
  5
\end{euleroutput}
\begin{eulercomment}
Menentukan tepi atas yang terbesar
\end{eulercomment}
\begin{eulerprompt}
>50+0.5
\end{eulerprompt}
\begin{euleroutput}
  50.5
\end{euleroutput}
\begin{eulerprompt}
>r=20.5:5:50.5; v=[1,8,10,5,4,2];
>T:=r[1:6]' | r[2:7]' | v'; writetable(T,labc=["TB","TA","frek"])
\end{eulerprompt}
\begin{euleroutput}
          TB        TA      frek
        20.5      25.5         1
        25.5      30.5         8
        30.5      35.5        10
        35.5      40.5         5
        40.5      45.5         4
        45.5      50.5         2
\end{euleroutput}
\begin{eulercomment}
Berdasarkan data, modus berada pada kelas 30.5-35.5.
\end{eulercomment}
\begin{eulerprompt}
>Tb=30.5, p=5, d1=2, d2=5
\end{eulerprompt}
\begin{euleroutput}
  30.5
  5
  2
  5
\end{euleroutput}
\begin{eulerprompt}
>Tb+p*d1/(d1+d2)
\end{eulerprompt}
\begin{euleroutput}
  31.929
\end{euleroutput}
\begin{eulercomment}
Jadi modus dari data hasil pengukuran berat badan 30 siswa di SDN 5
Jember adalah 31.92.

\end{eulercomment}
\eulersubheading{Contoh 8}
\begin{eulercomment}
Diketahui data sebagai berikut.\\
11,44,34,51,36,21,23,24,26,27,15,14,16,18,19,20,33,39,45,41,43\\
Tentukan kuartil, desil, dan persentil dari data tersebut!\\
Penyelesaian :
\end{eulercomment}
\begin{eulerprompt}
>data=[11,44,34,51,36,21,23,24,26,27,15,14,16,18,19,20,33,39,45,41,43];
>urut=sort(data)
\end{eulerprompt}
\begin{euleroutput}
  [11,  14,  15,  16,  18,  19,  20,  21,  23,  24,  26,  27,  33,  34,
  36,  39,  41,  43,  44,  45,  51]
\end{euleroutput}
\begin{eulerprompt}
>quartiles(data)
\end{eulerprompt}
\begin{euleroutput}
  [11,  18.5,  26,  40,  51]
\end{euleroutput}
\begin{eulercomment}
Dalam output hitung yang dihasilkan dari 'quartiles(data)' dapat
diketahui bahwa nilai Q1(kuartil bawah) = 18.5 , Q2(kuartil
tengah(median)) = 26, dan Q3(kuartil atas)= 40. Lalu untuk nilai
paling kanan dan paling kiri merupakan minimum dan maximum dari suatu
data yang diketahui.
\end{eulercomment}
\begin{eulerprompt}
>quantile(urut,0.1)
\end{eulerprompt}
\begin{euleroutput}
  15
\end{euleroutput}
\begin{eulercomment}
Dari hasil tersebut dapat diketahui bawha nilai desil ke-1 dan
persentil ke-10 adalah 15.

\end{eulercomment}
\eulersubheading{Contoh 9}
\begin{eulercomment}
Diketahui data sebagai berikut.\\
87,78,88,77,66,67,76\\
Tentukan varians dan simpangan baku dari data berikut.
\end{eulercomment}
\begin{eulerprompt}
>data=[87,78,88,77,66,67,76];
>urut=sort(data)
\end{eulerprompt}
\begin{euleroutput}
  [66,  67,  76,  77,  78,  87,  88]
\end{euleroutput}
\begin{eulerprompt}
>a=mean(urut)
\end{eulerprompt}
\begin{euleroutput}
  77
\end{euleroutput}
\begin{eulerprompt}
>dev=urut-a
\end{eulerprompt}
\begin{euleroutput}
  [-11,  -10,  -1,  0,  1,  10,  11]
\end{euleroutput}
\begin{eulerprompt}
>varians=mean(dev^2)
\end{eulerprompt}
\begin{euleroutput}
  63.429
\end{euleroutput}
\begin{eulerprompt}
>simpanganBaku= sqrt(varians)
\end{eulerprompt}
\begin{euleroutput}
  7.9642
\end{euleroutput}
\begin{eulercomment}
Jadi, variansnya adalah 63.42 dan simpangan bakunya adalah 7.96.
\end{eulercomment}
\end{eulernotebook}
\end{document}
